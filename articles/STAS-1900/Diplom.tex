\documentclass[12pt, a4paper]{article}
\usepackage[utf8]{inputenc}
\usepackage[russian]{babel}
\usepackage[T2A]{fontenc}
\usepackage{amsfonts}
\usepackage{amsmath}
\usepackage{indentfirst}
\usepackage{amsthm}
\usepackage{algorithm,algpseudocode}
\usepackage[hidelinks]{hyperref}
\DeclareMathOperator*{\minn}{min}
\DeclareMathOperator*{\argmin}{argmin}
\newtheorem{theorem}{Theorem}[section]
\newtheorem{state}{Утверждение}[section]
\newtheorem{lemma}{Лемма}[section]
\newtheorem{corollary}{Следствие}[section]


\usepackage[left=2cm,right=1.5cm,top=2cm,bottom=2cm]{geometry}
\linespread{1.25}

\usepackage{graphicx}
\graphicspath{{pictures/}}
\DeclareGraphicsExtensions{.pdf,.png,.jpg}

\begin{document}
\pagestyle{empty}

\begin{center}
	ФЕДЕРАЛЬНОЕ ГОСУДАРСТВЕННОЕ БЮДЖЕТНОЕ ОБРАЗОВАТЕЛЬНОЕ\\
	УЧРЕЖДЕНИЕ ВЫСШЕГО ОБРАЗОВАНИЯ\\
	<<МОСКОВСКИЙ ГОСУДАРСТВЕННЫЙ УНИВЕРСИТЕТ\\
	имени М.\,В.~ЛОМОНОСОВА>>
\end{center}
\vspace{4pt}
\begin{center}
	МЕХАНИКО-МАТЕМАТИЧЕСКИЙ ФАКУЛЬТЕТ
\end{center}
\vspace{4pt}
\begin{center}
	КАФЕДРА ВЫЧИСЛИТЕЛЬНОЙ МАТЕМАТИКИ
\end{center}
\vspace{1cm}
\begin{center}
	ВЫПУСКНАЯ КВАЛИФИКАЦИОННАЯ РАБОТА\\
	специалиста
\end{center}

\begin{center}
	\textbf{ОПТИМИЗАЦИЯ ТРАНСПОРТНОГО ПОТОКА \\
		    ПРИ ЗАДАННЫХ ПУНКТАХ ОТПРАВЛЕНИЯ И НАЗНАЧЕНИЯ \\
		    ВСЕХ УЧАСТНИКОВ ДВИЖЕНИЯ}
\end{center}
\vspace{1cm}
\begin{center}
	\begin{tabular}{p{9cm} l}
		& Выполнил студент $610$ группы\\
		& Пехтерев Станислав Игоревич\\
		& $\phantom{C_n^k=C_n^{n-k}}$\\
		& $\underline{\phantom{\int\limits_a^bf(x)dx=F(b)-F(a)}}$\\
		& подпись студента\\
		& $\phantom{\int\limits_f(z)dz=0}$\\
		& Научный руководитель:\\
		& доктор физико-математических наук \\
		& Васенин Валерий Александрович\\
		& $\phantom{C_n^k=C_n^{n-k}}$\\
		& $\underline{\phantom{\int\limits_a^bf(x)dx=F(b)-F(a)}}$\\
		& подпись научного руководителя\\
	\end{tabular}
\end{center}
\vspace{1cm}
\begin{center}
	Москва\\
	$2022$
\end{center}

\newpage
\pagestyle{plain}
\tableofcontents{}
\newpage	
	
 \section*{Тема}


\addcontentsline{toc}{section}{Введение}
\section*{Введение}


\newpage
\section{Постановка задачи}

Для начала поставим общую задачу оптимизации транспортного потока.

%{Пусть задан граф $G = (V, E)$, описывающий некоторую \textit{дорожную сеть}. Предположим, что имеется $n$ участников движения по этому графу. Каждый участник $i$ имеет точки отправления $A_i \in V$ и точки прибытия $B_i \in V$. Пусть множество $P_i$ есть множество всех простых путей из $A_i$ в $B_i$. Пусть декартово произведение ${P = \prod \limits_{i = 1} ^ n P_i}$ есть множество всех возможных комбинаций путей участников. Элементы этого множества назовем \textit{комбинацией путей}. Пусть известно, что при комбинации путей участников ${\textbf{p} \in P}$ $i$-ый участник затрачивает $T_i(\textbf{p})$ времени на передвижение. }

\subsection{Общая постановка задачи}

Пусть задан ориентированный граф $G = (V, E)$. Предположим, что имеется $n$ участников с заданными точками отправления $A_i \in V$ и прибытия $B_i \in V$. Пусть множество $P_i$ есть множество всех простых путей из $A_i$ в $B_i$. Элемент декартового произведения ${P = \prod \limits_{i = 1} ^ n P_i}$ назовем \textit{комбинацией путей}. Пусть известно, что при комбинации путей участников $\textbf{p} = \left(p_1, \ldots, p_n\right)\in P$ $i$-ый участник затрачивает $T_i(\textbf{p}) \in \mathbb{R}_{\ge 0}$ времени на свой путь. 
Функции $T_i$ назовем \textit{функциями временных затрат} участника $i$.
\textit{Некооперативным прокладыванием пути} в ориентированном графе $G$ назовем пятерку $F = (n, G, \{A_i\}_{i = 1}^{n}, \{B_i\}_{i = 1}^{n}, \{T_i\}_{i = 1}^{n})$. Некооперативное прокладывание пути предполагает, что каждый участник стремится сократить собственные временные затраты выбором пути $p_i$, несмотря на временные затраты других участников. 
Для того, чтобы скооперировать участников, введем некоторую функцию $\Phi (\textbf{p}) = \phi (T_1 (\textbf{p}), \ldots, T_n(\textbf{p}))$, определенную на множестве всех возможных комбинаций путей $P$ и отображающую его во множество действительных чисел. С помощью нее участники могут отслеживать, как влияет изменение их пути на общую картину движения. Такую функцию назовем \textit{функцией стоимости}.

Для заданных некооперативного прокладывания пути $F$ и функции стоимости $\Phi$ необходимо найти комбинацию путей $\textbf{p}^*$ такую, что функция стоимости на ней минимальна, то есть

\begin{equation}
	\label{eq:target_global_task_T} 
	\Phi (\textbf{p}^*) = \minn\limits_{ \textbf{p} \in P} \Phi (\textbf{p}).
\end{equation}

Комбинацию путей $\textbf{p}^*$ будем называть \textit {оптимальной}, а стоимость  $ \Phi (\textbf{p}^*)$ --- \textit{оптимальной стоимостью}.

Далее будем считать, что каждый участник имеет одинаковый приоритет в вопросе изменения своих временных затрат, то есть 

\begin{equation*}
	\frac{\partial \phi}{\partial T_i} \equiv 1, \text{ } i = 1, \ldots, n,
\end{equation*}
или
\begin{equation*}
\phi(T_1, \ldots, T_n) = \sum\limits_{i = 1}^nT_i.
\end{equation*}

Приведем ряд ограничений на функции $T_i (\textbf{p})$, которые позволят задать движения всех участников во времени при комбинации путей $\textbf{p}$.

\subsection{Постановка задачи в терминах модели движения}

Будем считать, что временные затраты участника на выбранном пути состоят из временных затрат на каждом ребре этого пути:


$$T_i (\textbf{p}) = \sum \limits_{e \in p_i} \overline{\tau}_{e, i} (\textbf{p}), $$
где функции $\overline{\tau}_{e, i} (\textbf{p})$ --- временные затраты $i$-ого участника на ребре $e$ при комбинации путей $\textbf{p}$. 


Для того, чтобы задать движение участника на пути, введем функции присутствия участника на ребре в момент времени $t$:
%%Будем рассматривать те некооперативные прокладывания пути, в которых
% существует движение каждого участника для каждой комбинации путей $\textbf{p}$, то есть 
%в каждый момент времени $t \in \mathbb{R}$ известно положение участника на пути. Таким образом, будем считать, что для каждой комбинации путей $\textbf{p}$ и времени $t$ известно присутствует ли участник $i$ на ребре $e$, то есть известны функции

$$
\theta_{e, i} (\textbf{p}, t) =
\begin{cases}
	1, & \text{если }  i\text{-ый участник движется по ребру $e$ в момент времени $t$,}  \\
	0, & \text{иначе},
\end{cases}
$$
где $\sum\limits_{e \in E} \theta_{e, i} (\textbf{p}, t)$ принимает значение $1$, пока $i$~-ый участник не посетит свою точку назначения $B_i$. Пусть достижение конца пути $p_i$ наступает в момент $T_i(\textbf{p})$, после чего $\sum\limits_{e \in E} \theta_{e, i} (\textbf{p}, t)$ принимает значение $0$. Получаем, что
	
\begin{equation}
	\label{eq:T_i_by_theta}
	T_i(\textbf{p}) = \sum \limits_{e \in p_i} \int\limits_{0}^{T_i(\textbf{p}) + \Delta t} \theta_{e, i} (\textbf{p}, t) dt, \: \forall \Delta t > 0.
\end{equation}

Будем считать, что движение каждого учаcтника является непрерывным и однонаправленным в графе $G$. Другими словами, участник не может резко повляться и исчезать на несмежных ребрах, а также находиться на уже пройденных ребрах. Таким образом, функции $\theta_{e, i} (\textbf{p}, t)$ являются индикаторами некоторых временных отрезков $[t_{e, i}^{in} (\textbf{p}), t_{e, i}^{out} (\textbf{p})]$, которые описывают однонаправленное движение:

\begin{equation}
	\label{eq:restr_t}
	\begin{cases}
		t_{e, i}^{in}(\textbf{p}), t_{e, i}^{out}(\textbf{p}) \in \mathbb{R}_+,  & i = 1, \dots, n, \text{ } e \in E, \\
		t_{e, i}^{in}(\textbf{p}) \le t_{e, i}^{out}(\textbf{p}), & i = 1, \dots, n, \text{ } e \in p_i,  \\
		t_{e, i}^{in}(\textbf{p}) = t_{e, i}^{out}(\textbf{p}) = 0, & i = 1, \dots, n, \text{ } e \notin p_i, \\
		t_{e_1, i}^{in} (\textbf{p}) = t_{e_2, i}^{out} (\textbf{p}), & i = 1, \dots, n, \text{ } e_1, e_2 \in p_i, \exists A, B, C \in V: e_1 = (A, B), e_2 = (B, C)\\
		t_{e, i}^{in} (\textbf{p}) = 0, & i = 1, \dots, n, \text{ } e = (A_i, X), X \in V.
	\end{cases}
\end{equation}

Заметим, что выбор таких отрезков пока неоднозначен. Далее считаем, что для каждого ребра $e$, участника $i$ и комбинации путей $\textbf{p}$ каким-то образом выбраны некоторые величины $t_{e, i}^{in}(\textbf{p}), t_{e, i}^{out}(\textbf{p})$, удовлетворяющие ограничениям \eqref{eq:restr_t}. Тогда функция временных затрат \eqref{eq:T_i_by_theta} $i$-ого участника примет вид

\begin{equation}
	\label{eq:T_i_by_t}
	T_i(\textbf{p}) = \sum \limits_{e \in E} t_{e, i}^{out}(\textbf{p}) - t_{e, i}^{in}(\textbf{p}).
\end{equation}

Функция стоимости в этом случае равна 

\begin{equation}
	\label{eq:target_func}
	\Phi(\textbf{p}) =\sum \limits_{i = 1}^n \sum \limits_{e \in E} t_{e, i}^{out}(\textbf{p}) - t_{e, i}^{in}(\textbf{p}).
\end{equation}

Считаем, что временные затраты участника $i$ на ребре $e$ ограничены некоторыми положительными константами $\overline{\tau}_{e, i}^{min}, \overline{\tau}_{e, i}^{max}$:

\begin{equation}
	\label{eq:add_restr}
		0 < \overline{\tau}_{e, i}^{min} \le t_{e, i}^{out}(\textbf{p}) - t_{e, i}^{in}(\textbf{p}) \le \overline{\tau}_{e, i}^{max}, e \in p_i,\, i = 1, \dots, n.
\end{equation}

Заметим, что задача оптимизации целевой функции \eqref{eq:target_func} с ограничениями \eqref{eq:restr_t}, \eqref{eq:add_restr} ставится в терминах задачи смешанного целочисленного линейного программирования с булевыми переменными $I_{e, i}$ и вещественными переменными $t_{e, i}^{in}, t_{e, i}^{out}$. Для участника $i$ первые отвечают факту проезда по ребру $e$, вторые --- моментам прохождения этого ребра. Однако в данных ограничениях решение уже есть --- участник $i$ передвигается по кратчайшему пути в графе $G$ с весами $\overline{\tau}_{e, i}^{min}$. Тривиальность решения связана с тем, что в данной задаче оптимизации отсутствуют влияния участников друг на друга. Для того, чтобы учесть это влияние, для каждого участника $i$ введем микроскопическую характеристику движения $v_i(\textbf{p}, t)$ ~--- положительную, ограниченную функцию, описывающую скорость участника. Также будем считать, что для каждого ребра $e \in E$ определена его длина $l_e > 0$.

Тогда имеет место следующее ограничение

\begin{equation}
	\label{eq:velocity_eq_by_theta}
	\int\limits_{0}^{T_i(\textbf{p})} \theta_{e, i} (\textbf{p}, t) v_i(\textbf{p}, t) dt = l_e, e \in p_i, i = 1, \dots, n,
\end{equation}
или
\begin{equation}
	\label{eq:velocity_eq_by_t}
	\int\limits_{t_{e, i}^{in}(\textbf{p})}^{t_{e, i}^{out}(\textbf{p})} v_i(\textbf{p}, t) dt = l_e, e \in p_i, i = 1, \dots, n.
\end{equation}

Будем говорить, что уравнения \eqref{eq:velocity_eq_by_t} задают \textit{движения участников}, а функции $v_i(\textbf{p}, t)$ назовем \textit{моделью движения}. Без ограничения общности считаем, что $\overline{\tau}_{e, i}^{min}, \overline{\tau}_{e, i}^{max}$ вычисляются в самом быстром и самом медленном вариантах передвижения по ребру $e$ участником $i$, а именно

\begin{equation}
	\label{eq:restr_add_concrete}
	\overline{\tau}_{e, i}^{min} = \frac{l_e}{\max\limits_{\textbf{p} \in P, t \in \mathbb{R}} \left(  v_i(\textbf{p}, t) \right)}, \; \overline{\tau}_{e, i}^{max} = \frac{l_e}{\min\limits_{\textbf{p} \in P, t \in \mathbb{R}} \left(  v_i(\textbf{p}, t) \right)}.
\end{equation}

Заметим, что величины $t_{e, i}^{in}(\textbf{p}), t_{e, i}^{out}(\textbf{p}) \in \mathbb{R}_+$ --- произвольные вещественные величины, которые удовлетворяют ограничениям \eqref{eq:restr_t}, \eqref{eq:add_restr}, \eqref{eq:velocity_eq_by_t}, \eqref{eq:restr_add_concrete}.

\begin{state}
\label{state:modeling}
Пусть задан ориентированный граф $G$ с положительными длинами $\{l_e\}_{e \in E}$, модель движения $v_i(\textbf{p}, t)$, и для каждого ребра $e$, участника $i$ и комбинации путей $\textbf{p}$ задано множество величин $t_{e, i}^{in}(\textbf{p}), t_{e, i}^{out}(\textbf{p}) \in \mathbb{R}_+$, для которых выполняются ограничения \eqref{eq:restr_t}, \eqref{eq:add_restr}, \eqref{eq:velocity_eq_by_t}, \eqref{eq:restr_add_concrete}. Тогда $t_{e, i}^{out}(\textbf{p})$ и $t_{e, i}^{in}(\textbf{p})$, $e \in p_i$ есть функции от комбинации путей $\textbf{p} \in P$.
\end{state}
\begin{proof}
Зафиксируем некоторую комбинацию путей $\textbf{p}$. Опишем алгоритм поиска значений $t_{e, i}^{out}(\textbf{p})$ и $t_{e, i}^{in}(\textbf{p})$ и покажем его корректность.

\begin{algorithm}[H]
	\caption{Моделирование движения участников}
	\label{alg:modeling}
	{\bf {Input:}} количество участников $n$, ориентированный граф $G$, комбинация путей $\textbf{p}$ графа $G$\\
	{\bf {Output:}} $t_{e, i}^{out}(\textbf{p})$, $t_{e, i}^{in}(\textbf{p})$, $e \in p_i, i = 1, \ldots, n$\\
	{\bf {Data:}} текущее время $t$, текущее ребро $e_i$ и часть пройденного ребра $x_i$ участника $i$
	\begin{algorithmic}[1]
		\State $t = 0$
		\For{$i = 1, \ldots, n$}
		\State $e_i \gets$ { первое ребро пути $p_i$}
		\State $x_i \gets 0$
		\State $t_{e_i, i}^{in}(\textbf{p}) \gets 0$ 
		\EndFor
		\While{$\exists i: i \text { --- не доехал}$}
		\State $\tau^* \gets \argmin\{ \tau \in \mathbb{R}: \tau > t, \int\limits_{t}^{\tau} v_i(\textbf{p}, t) dt = (1 - x_i) l_{e_i}, i \text{--- не доехал}  \}_{i = 1}^n$
		\For{$i = 1, \ldots, n$ \textbf{and} $i - \text{ не доехал}$ }
		\State $x_i \gets x_i + \frac{1}{l_{e_i}} \int\limits_{t}^{\tau^*} v_i(\textbf{p}, t) dt$
			\If{$x_i = 1$ \textbf{and} $e_i$ - не последнее ребро пути $p_i$ }
				\State $x_i \gets 0$
				\State $t_{e_i, i}^{out}(\textbf{p}) \gets \tau^*$ 
				\State $e_i \gets$ следующее ребро за $e_i$ в пути $p_i$
				\State $t_{e_i, i}^{in}(\textbf{p}) \gets \tau^*$ 
			\EndIf
		\EndFor
		\State $t \gets \tau^*$
		\EndWhile
	\end{algorithmic}
\end{algorithm}


Описанный алгоритм называется \textit{моделированием движения}.

\textit{Корректность}. Для доказательства корректности алгоритма достаточно доказать корректность шага 8 и достижимость шага 11. Это следует из того, что функция скорости ограничена снизу (см. ограничения \eqref{eq:add_restr}, \eqref{eq:restr_add_concrete}). Алгоритм сойдется, поскольку пути $p_i$ конечны.

\end{proof}
 Используя это утверждение, можем ввести следующее понятие:

\textit{Некооперативным передвижением} по графу $G$ c положительными длинами $\{l_e\}_{e \in E}$ в модели движения $v_i(\textbf{p}, t)$ назовем такое некооперативное прокладывание пути \\$F = \left(n, G, \{A_i\}_{i = 1}^{n}, \{B_i\}_{i = 1}^{n}, \left\{\sum\limits_{e \in p_i} t_{e, i}^{out}(\textbf{p}) - t_{e, i}^{in}(\textbf{p})\right\}_{i = 1}^{n}\right)$, где функции $t_{e, i}^{in}(\textbf{p}), t_{e, i}^{out}(\textbf{p})$ получены путем моделирования движения с моделью движения $v_i(\textbf{p}, t)$ и длинами ребер $\{l_e\}_{e \in E}$ графа $G$.
Значит, постановка задачи в терминах модели движения следующая:

Пусть задано некооперативное передвижение по графу $G$ c положительными длинами $\{l_e\}_{e \in E}$ в модели движения $v_i(\textbf{p}, t)$.
Требуется найти комбинацию путей $\textbf{p}$ такую, что функция 

\begin{equation}
\label{eq:target_task_end}
\Phi(\textbf{p}) =\sum \limits_{i = 1}^n \sum \limits_{e \in E} t_{e, i}^{out}(\textbf{p}) - t_{e, i}^{in}(\textbf{p})
\end{equation}
минимальна.

Оказывается, что для любого некооперативного прокладывания пути при любых положительных длинах $\{l_e\}_{e \in E}$ существует эквивалентное ему некооперативное передвижение в графе с этими длинами в некоторой модели движения $v_i(\textbf{p}, t)$. Другими словами, любое некооперативное прокладывание пути можно промоделировать.

\newpage
\section{Модели движения}

\label{sec:models}

\begin{state}
	\label{state:eqv}
	Пусть заданы некоторое некооперативное прокладывание пути $F$ и положительные длины ребер  $\{l_e\}_{e \in E}$ графа $G$. Тогда можно задать модель движения $v_i(\textbf{p}, t)$ такую, что затраченное время на передвижение $i$-ым участником при комбинации путей $\textbf{p}$ совпадает с его временными затратами, то есть выполняется \eqref{eq:T_i_by_t}.
	
\end{state}

\begin{proof}
Рассмотрим модель движения с постоянными скоростями  
$$v_i(\textbf{p}, t) = \overline{v}_i(\textbf{p}) = \frac{T_i (\textbf{p})}{\sum \limits_{e \in p_i} l_e}.$$

Промоделировав движение с такими скоростями, получим \eqref{eq:T_i_by_t}.
\end{proof}

Таким образом, можно сказать, что каждый выбор комбинации путей $\textbf{p}$ можно промоделировать.

Очевидно, что решение задачи перебором не является практичным --- оно сводится к перебору всех комбинаций путей $\textbf{p} \in P$. Так, например, количество таких комбинаций в полном графе составляет $2^{n (|V| - 1)}$, перебрать которые в условиях реальных данных вычислительно сложно.
Однако в случае, когда условие \eqref{eq:velocity_eq_by_t} можно описать в терминах задачи удовлетворения ограничений, задача оптимизации \eqref{eq:target_task_end} может быть описана в терминах смешанного целочисленного линейного программирования и, как следствие, может быть решена стандартным решателем. Оказывается, можно выделить целый класс таких моделей движения, для которых это возможно.

\subsection{Макроскопические модели}

Предположим, что скорость участника зависит от некоторой общей для участников величины. Например, от функции \textit{загруженности ребра}

$$ n_{e}(\textbf{p}, t) = \sum\limits_{i = 1}^n\theta_{e, i}(\textbf{p}, t),$$
значение которой в момент времени $t$ соответствует количеству участников на ребре $e$ при комбинации путей $\textbf{p}$. Предположим, скорость участника зависит только от загруженности ребра, на котором он находится в момент времени $t$, то есть существует ограниченная функция $v : \{0, 1, \dots, n\} \rightarrow \mathbb{R}_{> 0}$ такая, что

\begin{equation}
	\label{eq:velocity_eq_macro}
	 v_i(\textbf{p}, t) = \sum \limits _{e \in E} \theta_{e, i} (\textbf{p}, t) v (n_e (\textbf{p}, t)), \; i = 1, \dots, n
\end{equation}

Такую модель движения в дальнейшем будем называть \textit{макроскопической}.
Например, естественно рассмотреть модель $ v (n_e (\textbf{p}, t)) = \frac{v_{max}}{n_e (\textbf{p}, t)}$. В общем случае такая модель задается последовательностью значений  $\{v(k)\}_{k = 1}^n$.

\begin{lemma}
	\label{lemma:lt}
	Пусть даны вещественные переменные $a$, $b$ целочисленного программирования, и известно, что существует константа $M > 0$: $|a| < M$, $|b| < M$. Тогда можно добавить новую целочисленную переменную $\textbf{1} (\{a < b\}) \in \{0, 1\}$ такую, что
	
	\begin{equation*}
		\textbf{1} (\{a < b\}) = 
		\begin{cases}
			1,\, a < b,
			\\
			0,\, a \ge b.
		\end{cases}
	\end{equation*}

\end{lemma}

\begin{proof}
	Добавим в нашу задачу два неравенства:
	
	$$ 2M (\textbf{1} (\{a < b\}) - 1) < b - a \le 2M\textbf{1} (\{a < b\}) $$
	
	Очевидная проверка показывает, что неравенство выполняется для любых $a, b$.
	
	
\end{proof}

\begin{state}
	
	\label{state:lin_prog}
	
	Пусть модель движения $ v_i(\textbf{p}, t)$ макроскопическая. Тогда задача \eqref{eq:target_task_end} есть задача смешанного целочисленного линейного программирования.
\end{state}

\begin{proof}
	Докажем для случая $n = 2$. Для случаев $n \ge 2$ доказательство аналогичное.
	
	Пусть имеется задача смешанного целочисленного линейного программирования \eqref{eq:restr_t} с переменными $t_{e, i}^{in}, t_{e, i}^{out}, I_{e, i}, e \in E, i = 1, 2$. Преобразуем условие \eqref{eq:velocity_eq_by_t} к каноническому виду задачи удовлетворения ограничений. Для удобства обозначим обоих участников индексами $i, j \in \{1, 2\}$.
	
	$$\int\limits_{0}^{\infty} \theta_{e, i} (\textbf{p}, t) v_i(\textbf{p}, t)dt = \int\limits_{0}^{\infty} \theta_{e, i} (\textbf{p}, t) \sum \limits _{e^1 \in E} \theta_{e^1, i} (\textbf{p}, t) v (n_{e^1} (\textbf{p}, t)) dt = $$
	
	$$\int\limits_{0}^{\infty} \theta_{e, i} (\textbf{p}, t)  v (n_{e} (\textbf{p}, t)) dt = 
	  \int\limits_{ \substack{n_{e} (\textbf{p}, t) = 1}} \theta_{e, i} (\textbf{p}, t)  v (n_{e} (\textbf{p}, t)) dt +
	  \int\limits_{ \substack{n_{e} (\textbf{p}, t) = 2}} \theta_{e, i} (\textbf{p}, t)  v (n_{e} (\textbf{p}, t)) dt = $$
	
	$$\int\limits_{ \substack{\theta_{e, j} (\textbf{p}, t) = 0}} \theta_{e, i} (\textbf{p}, t)  v (1) dt +
	  \int\limits_{ \substack{\theta_{e, j} (\textbf{p}, t) = 1}} \theta_{e, i} (\textbf{p}, t)  v (2) dt = $$
	  
    $$\int\limits_{0}^{\infty} \theta_{e, i} (\textbf{p}, t)  v (1) dt - 
      \int\limits_{ \substack{\theta_{e, j} (\textbf{p}, t) = 1}} \theta_{e, i} (\textbf{p}, t)  v (1) dt +
	  \int\limits_{ \substack{\theta_{e, j} (\textbf{p}, t) = 1}} \theta_{e, i} (\textbf{p}, t)  v (2) dt = $$
	  
	$$v (1) \int\limits_{0}^{\infty} \theta_{e, i} (\textbf{p}, t) dt +
	  (v (2) - v(1)) \int\limits_{0}^{\infty} \theta_{e, i} (\textbf{p}, t) \theta_{e, j} (\textbf{p}, t) dt = $$
	  
    $$v (1) \overline{\tau}_{e, i} (\textbf{p}) +
    (v (2) - v(1)) \int\limits_{0}^{\infty} \theta_{e, i} (\textbf{p}, t) \theta_{e, j} (\textbf{p}, t) dt = l_e, e \in p_i$$
	  
	Неизвестный интеграл --- время совместного проезда участников на ребре $e$.
	  
	В переменных задачи смешанного целочисленного программирования получим:
	
	$$v(1) (t_{e, i}^{out} - t_{e, i}^{in}) + (v(2) - v(1)) (t_{e, ij}^{out} - t_{e, ij}^{in}) = l_e I_{e, i},$$
	где новые переменные $t_{e, ij}^{in}$, $t_{e, ij}^{out}$ отвечают началу и концу совместного проезда участников. Иначе говоря, $[t_{e, ij}^{in}, t_{e, ij}^{out}] = [t_{e, i}^{in}, t_{e, i}^{out}] \cap [t_{e, j}^{in}, t_{e, j}^{out}]$. Просуммировав по всем ребрам $e \in E$, получим
	
	$$v(1) \sum \limits _{e \in E} (t_{e, i}^{out} - t_{e, i}^{in}) = \sum \limits _{e \in E} l_e I_{e, i} - (v(2) - v(1)) \sum \limits _{e \in E} (t_{e, ij}^{out} - t_{e, ij}^{in}).$$
	
	Заметим, что левая часть представляет собой временные затраты участника $i$ с коэффициентом $v(1)$, поэтому задачу оптимизации можно переписать в виде
	$$ \frac{1}{v (1)} \sum\limits_{i = 1}^n \sum \limits _{e \in E} l_e I_{e, i} + \frac{v(1) - v(2)}{v (1)}  \sum\limits_{i = 1}^n \sum \limits _{e \in E} (t_{e, ij}^{out} - t_{e, ij}^{in}) \rightarrow \minn .$$
	
	Для завершения доказательства необходимо показать, что переменные $t_{e, ij}^{in}, t_{e, ij}^{out}$ описываются линейными ограничениями. Обозначим
	\begin{align*}
		& \Delta t = t_{e, ij}^{out} - t_{e, ij}^{in} \\
		& \Delta t_1 =  t_{e, i}^{out} - t_{e, i}^{in}  \\
		& \Delta t_2 =  t_{e, j}^{out} - t_{e, j}^{in} \\
		& \Delta t_3 =  t_{e, i}^{out} - t_{e, j}^{in} \\
		& \Delta t_4 =  t_{e, j}^{out} - t_{e, i}^{in}.
	\end{align*}
	Используя лемму \ref{lemma:lt} при $M = \max\limits_{e \in E, k = i, j} \overline{\tau}_{e, k}^{max}$, добавим в задачу новые переменные $\textbf{1} (\{ \Delta t_k > \Delta t_l\}), \; k \ne l, \; k, l \in \{1, 2, 3, 4\}$. Рассмотрим величину $T_{max} = |E|M$. В случае $v(1) \ge v(2)$ добавим в нашу задачу следующие неравенства: 
	\begin{align*}
	& \Delta t \ge 0, \\
	& \Delta t \ge \Delta t_k - T_{max} \sum \limits_{l \ne k} {\textbf{1} (\{ \Delta t_k > \Delta t_l\})}, \; k = 1, 2, 3, 4.
	\end{align*}
	В случае $v(1) < v(2)$ добавим те же ограничения с другим знаком неравенства. Тогда с учетом оптимизации переменная $\Delta t$ есть длина отрезка $[t_{e, ij}^{in}, t_{e, ij}^{out}]$. 
	
\end{proof}

\begin{corollary}
	\label{corollary:rel}

	Пусть модель движения $ v_i(\textbf{p}, t) = \sum \limits _{e \in E} \theta_{e, i} (\textbf{p}, t) v (n_e (\textbf{p}, t))$ макроскопическая и последовательность $v(n) > 0, \forall n \in \mathbb{Z}_+$ убывает. Предположим, что оптимальное время движения в модели c постоянной скоростью $v(1)$ есть $\widetilde{T}$. Тогда

	$$ \widetilde{T} \le T \le \frac {v(1)}{v(n)} \widetilde{T}.$$
	
\end{corollary}

\begin{proof}
	Докажем каждое неравенство по отдельности
	
	1. В модели, где все участники едут с постоянными скоростями, движение происходит по кратчайшим путям. Тогда временные затраты есть $\widetilde{T} = \frac{1}{v(1)} \sum \limits _{i = 1} ^ n \sum\limits_{e \in p_i} l_e$, где $p_i$ - кратчайшие пути.
	На тех же путях задается худший случай макроскопической модели --- все едут с минимальной скоростью, то есть $ T = \frac{1}{v(n)} \sum \limits _{i = 1} ^ n \sum\limits_{e \in p_i} l_e$. Тогда получим
	
	$$T \le  \frac{1}{v(n)} \sum \limits _{i = 1} ^ n \sum\limits_{e \in p_i} l_e = \frac {v(1)}{v(n)} \widetilde{T}.$$
	
	2. Производя аналогичные вычисления, что и в доказательстве \ref{state:lin_prog}, получаем, что функция оптимизации имеет вид
	
	$$ \frac{1}{v (1)} \sum\limits_{i = 1}^n \sum \limits _{e \in E} l_e I_{e, i} +  \sum\limits_{k = 2}^{n} \frac{v(1) - v(k)}{v (1)}  \sum\limits_{i = 1}^n \sum \limits _{e \in E} \sum\limits _{\substack{ s_k \in 2^n \\ |s_k| = k}}  \Delta t_{e, s_k} \rightarrow \minn ,$$
	где переменные $\Delta t_{e, s_k}$ отвечают времени совместного движения участников $s_k$  (и только их) по ребру $e$.
	
	Тогда получим
		$$T \ge 
		  \minn \left(  \frac{1}{v (1)} \sum\limits_{i = 1}^n \sum \limits _{e \in E} l_e I_{e, i} \right) 
		+ \minn \left(  \sum\limits_{k = 2}^{n} \frac{v(1) - v(k)}{v (1)}  \sum\limits_{i = 1}^n \sum \limits _{e \in E} \sum\limits _{\substack{ s_k \in 2^n \\ |s_k| = k}}  \Delta t_{e, s_k} \right) \ge \widetilde{T}.$$

\end{proof}

Таким образом, мы получили класс моделей движения, для которых задача оптимизации транспортного потока может быть поставлена в терминах смешанного целочисленного линейного программирования. Однако такой класс моделей движения плохо описывает реальное движение автомобилей. Так, например, модель не учитывает расстояние между участниками и их порядок на ребре.

\subsection{Микроскопические модели}
\textit {Микроскопическими} называются модели движения, которые не являются макроскопическими, то есть не представимы в виде \eqref{eq:velocity_eq_macro}. В таких моделях явно исследуется движение каждого автомобиля.
Выбор такой модели позволяет теоретически достичь более точного описания движения автомобилей по сравнению с макроскопической моделью, однако на практике этот подход требует больших вычислительных ресурсов.

В качестве примера рассмотрим движение по бесконечному ребру. Пусть ${x_i(t) \in [0, +\infty)}$~--- координаты участника $i$. Предположим, что скорости участников ограничены некоторой общей величиной $v_{max}$. Пусть в момент времени ${t = 0}$ выполняется $x_1(0) \le x_2(0) \le \dots \le x_n(0)$.

\subsubsection*{Модель пропорциональной скорости}
Рассмотрим модель, в которой скорость участника пропорциональна расстоянию до впереди идущего участника.
Положим $d_{i} (t) = x_{i + 1} (t) - x_{i} (t), \; i = 1, \dots, n - 1$.
Без ограничения общности считаем, что $d_{i} (0) < D, \; i = 1, \dots, n - 1$, где $D$ --- расстояние, на котором происходит взаимодействие участников. Иначе рассмотрим подпоследовательности участников, для которых выполняется это условие.

Пусть модель движения есть
\begin{equation}
	\label{eq:micro}
	v_i(t)=
	\begin{cases}
		v_{max}, & i = n,
		\\
		v_{max} \frac{d_i(t)}{D} ,& i \ne n.
	\end{cases}
\end{equation}

Для поиска функций $x_i(t)$ достаточно рассмотреть систему дифференциальных уравнений

$$ \dot{d_i} (t) = v_{i + 1} (t) - v_i (t).$$

Решением такой системы является

$$d_{n - k} (\tau) = \sum \limits_{l = 0} ^ {k - 1} \left(\frac{d_{n - k + l} (0) - D}{l!} \tau^l e ^ {-\tau}\right) + D ,$$
где ${\tau = \frac{v_{max}}{D}t}$. Модель обладает тем свойством, что порядок участников постоянен и участники не покидают зону взаимодействия $D$. 

Данная модель хорошо описывает реальное движение участников, однако ее практическое применение вызывает сложности, поскольку решение уравнения, вычисляемое на шаге 2 моделирования движения (см. алгоритм \ref{alg:modeling}), может быть найдено только приближенно.

\subsubsection*{Модель снижения скорости}

Предположим, что существует некоторая величина $c_n$, которая отвечает за последовательное снижение скорости участников относительно их порядка:

$$v_{n - k} = v_{max} - c_n k, \quad k = 0, \dots, n - 1.$$
Величину $c_n$ выберем из соображений, что $v_0 = \frac{v_{max}}{n}$. Тогда $c_n = \frac{v_{max}}{n}$. Если смоделировать данное движение на графе, то функции скоростей будут кусочно--постоянными. Это связано с тем, что при смене ребра некоторым участником меняются порядок и величина $n_e(\textbf{p}, t)$. Поэтому модель снижения скорости не лучшим образом описывает реальное движение, однако проста в использовании.

Пока для микроскопических моделей нет очевидного подхода к решению. Однако для любой модели движения можно описать алгоритмы оптимизации, которые сходятся к <<локальному минимуму>>. Рассмотрим такие алгоритмы в следующем разделе.

\newpage
\section{Равновесие транспортных потоков}
\label{sec:rovn}
В этом разделе мы исследуем задачу поиска равновесия транспортных потоков как возможность поиска оптимального транспортного потока.

\subsection{Некооперативное и кооперативное равновесие}

\textit{Некооперативной игрой в нормальной форме} назовем тройку $\Gamma = (n, \{S_i\}_{i = 1}^n, \{H_i\}_{i = 1}^n)$, где $n \in \mathbb{N}$ --- количество участников игры, $S_i$ --- множество стратегий участника $i \in {1, \dots, n}$, $H_i$ --- функция выигрыша участника $i$, определенная на множестве ситуаций $S = \prod\limits_{i = 1}^n S_i$ и отображающая его во множество действительных чисел.

\textit{Равновесием Нэша} некооперативной игры в нормальной форме $\Gamma = (n, \{S_i\}_{i = 1}^n, \{H_i\}_{i = 1}^n)$ назовем стратегию $\textbf{s}^* = (s^*_1,\dots, s^*_n) \in S$ такую, что изменение своей стратегии с $s_i^*$ на любую другую $s \in S_i$ невыгодно ни одному игроку $i$. В наших обозначениях равновесие Нэша принимает вид

$$H_i(\textbf{s}^*) \ge H_i(\left(s^*_1, \ldots, s^*_{i - 1}, s, s^*_{i + 1}, \ldots, s^*_{n} \right)), \; \forall s \in S_i, \; \, i = 1, \dots, n. $$ 
Заметим, что в общем случае ничего нельзя сказать о существовании и единственности равновесия некооперативной игры.

Введем понятия некооперативного и кооперативного равновесия, которые являются равновесиями Нэша в терминах некооперативного прокладывания пути, где выигрыш заключается в сэкономленном времени передвижения и стоимости соответственно.

\textit{Некооперативным равновесием} некооперативного прокладывания пути $F$ назовем комбинацию путей $\widehat{\textbf{p}} \in P$, которая является равновесием Нэша некооперативной игры $\widehat{\Gamma} = (n, \{P_i\}_{i = 1}^n, \{-T_i\}_{i = 1}^n)$. Множество всех некооперативных равновесий обозначим $\widehat{P}$.

\textit{Кооперативным равновесием} некооперативного прокладывания пути $F$ и функции стоимости $\Phi (\textbf{p})$ назовем комбинацию путей $\widetilde{\textbf{p}} \in P$, которая является равновесием Нэша некооперативной игры $\widetilde{\Gamma} = (n, \{P_i\}_{i = 1}^n, \{-\Phi\}_{i = 1}^n)$. Множество всех кооперативных равновесий обозначим $\widetilde{P}$.

Заметим, что определения некооперативного прокладывания пути и некооперативной игры эквивалентны.  
Таким образом, любой пример игры, где равновесия Нэша не существует, можно использовать как пример некооперативного передвижения по графу, где нет кооперативного равновесия.
Однако для кооперативного равновесия верно обратное:

\begin{state}
Множество кооперативных равновесий $\widetilde{P}$ не пусто, причем
оптимальная комбинация путей является таким равновесием, то есть $\textbf{p}^* \in \widetilde{P}$.
\end{state}

\begin{proof}
	Поскольку для любого $\textbf{p} \in P$ 
	$$\Phi (\textbf{p}^*) \le \Phi (\textbf{p}),$$
	то неравенство верно и для комбинаций путей $\textbf{p} = \left(p^*_1, \ldots, p^*_{i - 1}, p, p^*_{i + 1}, \ldots, p^*_{n} \right), \text{ } p \in P_i, \text{ } i = 1, \ldots, n$.
\end{proof}

В некотором смысле кооперативным равновесием можно назвать <<локальным минимум>> функции $\Phi$.

\subsection{Поиск кооперативного равновесия}
Рассмотрим ряд алгоритмов, позволяющих получить некоторое кооперативное равновесие.

Общим свойством всех этих алгоритмов является предположение о том, что существует некоторый алгоритм $\alpha (\{\Phi_i\}_{i = 1}^n)$, позволяющий решить задачу оптимизации некоторой функции стоимости $\Phi_i (\textbf{p}) = \phi_i(T_1(\textbf{p}), \ldots, T_n(\textbf{p}))$ посредством выбора пути $p_i$. В работе Л.\,Е.~Разумовой~\cite{Luba} представлен один из таких алгоритмов построения оптимального пути $p_i$ за полиномиальное относительно входных данных время при условии, что функция $\Phi_i (\textbf{p})$ удовлетворяет неравенству

\begin{equation}
	\label{eq:luba_1}
	\Phi_i (p_1, \ldots, p_{i - 1}, pe, p_{i + 1}, \ldots, p_n) \le 
	\Phi_i (p_1, \ldots, p_{i - 1}, qe, p_{i + 1}, \ldots, p_n),
\end{equation}
где $p, q$ --- два пути к некоторой вершине $B \in V$, ребро $e$ выходит из этой вершины и путь $p$ <<дешевле>>, чем $q$ относительно стоимости $\Phi_i$:

\begin{equation}
	\label{eq:luba_2}
	\Phi_i (p_1, \ldots, p_{i - 1}, p, p_{i + 1}, \ldots, p_n) \le
  	\Phi_i (p_1, \ldots, p_{i - 1}, q, p_{i + 1}, \ldots, p_n).
\end{equation}

Таким образом, имея некоторые функции стоимости $\Phi_i (\textbf{p})$, удовлетворяющие условиям \eqref{eq:luba_1}, \eqref{eq:luba_2}, можем описать полиномиальный алгоритм $\beta$, позволяющий перейти к меньшей стоимости передвижения путем изменения некоторого пути  $p_i$ участника $i$. Для поиска кооперативного рановесия достаточно найти неподвижную точку алгоритма $\beta$. 

\begin{algorithm}[H]
	\caption{Поиск неподвижной точки алгоритма $\beta$}
	\label{alg:coop_find1}
	{\bf {Input:}} Начальная комбинация путей $\textbf{p}_0 \in P$, алгоритм $\beta$, количество итераций $iter$\\
	{\bf {Output:}} кооперативное равновесие $\widetilde{\textbf{p}} \in \widetilde{P}$\\
	{\bf {Data:}} $\textbf{p}_{cur}$ - текущая комбинация путей, $\textbf{p}_{new}$ - новая комбинация путей, $i$ - номер итерации
	\begin{algorithmic}[1]
		\State $\textbf{p}_{cur} \gets \textbf{p}_0$
		\State $i \gets 0$
		\While{$i < iter$}
		\State $\textbf{p}_{new} \gets \beta (\textbf{p}_{cur}) $
		\State $i \gets i + 1$
		\If{$\textbf{p}_{new}$ = $\textbf{p}_{cur}$}
			\State \textbf{return $\textbf{p}_{cur}$}
		\EndIf
		\EndWhile
	\State \textbf{return $\textbf{p}_{cur}$}
	\end{algorithmic}
\end{algorithm}

Данный алгоритм не дает гарантий, что сходимость произойдет за число итераций, не зависящее от количества комбинаций путей. Однако результатом каждой итерации алгоритма $\beta$ является новая комбинация путей $\textbf{p}$ меньшей стоимости относительно $\Phi$. 

%Опишем алгоритм, сложность которого не зависит от количества комбинаций путей $|P|$ и, который находит оптимальную комбинацию путей $\textbf{p}$, 
Опишем алгоритм, который с некоторыми допущениями на модель движения имеет сложность, не зависищую от количества комбинаций путей $|P|$, и находит оптимальную комбинацию путей $\textbf{p}$.
Также алгоритм не зависит от начальной комбинации путей $\textbf{p}_0 \in P$. Предположим, имеется набор функций $\{\{\Phi_{i, k} \}_{i = 1}^k\}_{k = 1}^n$, для каждого $k$ отображающие декартово прозведение $\prod\limits_{i = 1}^kP_i$ во множество действительных чисел $\mathbb{R}$. Считаем, что все функции удовлетворяют условиям \eqref{eq:luba_1}, \eqref{eq:luba_2}.


\begin{algorithm}[H]
	\caption{Последовательное добавление участников в движение}
	\label{alg:coop_find2}
	{\bf {Input:}} алгоритм $\alpha$, алгоритмы $\{\beta_k\}_{k = 1}^n$\\
	{\bf {Output:}} кооперативное равновесие $\widetilde{\textbf{p}} \in \widetilde{P}$\\
	{\bf {Data:}} $\textbf{p}_{k} \in \prod\limits_{i = 1}^kP_i$ - кооперативное отношение для первых $k$ участников, $\textbf{p}_{new}$ - новая комбинация путей, $k$ - номер итерации
	\begin{algorithmic}[1]
		\State $k \gets 0$
		\While{$k <= n$}
		\State $\textbf{p}_{k + 1} \gets (\textbf{p}_{k}, \alpha (\textbf{p}_{k}))$
		\State Запустим алгоритм \ref {alg:coop_find1} на комбинации путей $\textbf{p}_{k + 1}$
		\State $k \gets k + 1$
		\EndWhile
	\end{algorithmic}
\end{algorithm}

При наложенном на модель движения условии, что добавление оптимального пути участника--эгоиста не меняет свойства оптимальности итоговой комбинации путей, можно сказать, что алгоритм сходится к оптимальной комбинации путей. Для того, чтобы алгоритм сошелся за $n$ применений алгоритма $\alpha$, достаточно изменить условие оптимальности на условие кооперативного равновесия.

%Заметим, что описанные в разделе \ref{sec:models} модели удовлетворяют условиям, описанным выше при функциях стоимости $\Phi_i (\textbf{p}) = T_i (\textbf{p})$.

\newpage
\section{Результаты}

\newpage
\section{Заключение}

    \newpage
\begin{thebibliography}{0}
	
	\addcontentsline{toc}{section}{Литература}

	\bibitem{Luba} \textit{Л.\,Е.~Разумова, С.\,А.~Афонин},  ``Построение оптимального маршрута при заданной модели движения других участников движения''.
	
	\bibitem{Gluts} \textit{А.\,А.~Глуцюк}, ``О двумерных полиномиально интегрируемых бильярдах на поверхностях постоянной кривизны'', Доклады Академии наук, \textbf{481}, 6, 2018, 594--598
	
	\bibitem{MirByal} \textit{M.~Bialy, A.\,E.~Mironov}, ``Angular billiard and algebraic Birkhoff conjecture'', Adv. Math., \textbf{313}, 2017, 102--126
	
\end{thebibliography} 

\end{document}