\documentclass[12pt, a4paper]{article}
\usepackage[utf8]{inputenc}
\usepackage[russian]{babel}
\usepackage[T2A]{fontenc}
\usepackage{amsfonts}
\usepackage{amsmath}
\usepackage{indentfirst}
\usepackage{amsthm}
\usepackage{algorithm,algpseudocode}
\usepackage{comment}
\DeclareMathOperator*{\minn}{min}
\DeclareMathOperator*{\argmin}{argmin}
\newtheorem{theorem}{Theorem}[section]
\newtheorem{state}{Утверждение}[section]
\newtheorem{lemma}{Лемма}[section]
\newtheorem{corollary}{Следствие}[section]


\usepackage[left=2cm,right=1.5cm,top=2cm,bottom=2cm]{geometry}
\linespread{1.25}

\usepackage{graphicx}
\graphicspath{{pictures/}}
\DeclareGraphicsExtensions{.pdf,.png,.jpg}

\begin{document}
\pagestyle{empty}

\begin{center}
	ФЕДЕРАЛЬНОЕ ГОСУДАРСТВЕННОЕ БЮДЖЕТНОЕ ОБРАЗОВАТЕЛЬНОЕ\\
	УЧРЕЖДЕНИЕ ВЫСШЕГО ОБРАЗОВАНИЯ\\
	<<МОСКОВСКИЙ ГОСУДАРСТВЕННЫЙ УНИВЕРСИТЕТ\\
	имени М.\,В.~ЛОМОНОСОВА>>
\end{center}
\vspace{4pt}
\begin{center}
	МЕХАНИКО-МАТЕМАТИЧЕСКИЙ ФАКУЛЬТЕТ
\end{center}
\vspace{4pt}
\begin{center}
	КАФЕДРА ВЫЧИСЛИТЕЛЬНОЙ МАТЕМАТИКИ
\end{center}
\vspace{1cm}
\begin{center}
	ВЫПУСКНАЯ КВАЛИФИКАЦИОННАЯ РАБОТА\\
	специалиста
\end{center}

\begin{center}
	\textbf{ОПТИМИЗАЦИЯ ТРАНСПОРТНОГО ПОТОКА \\
		    ПРИ ЗАДАННЫХ ПУНКТАХ ОТПРАВЛЕНИЯ И НАЗНАЧЕНИЯ \\
		    ВСЕХ УЧАСТНИКОВ ДВИЖЕНИЯ}
\end{center}
\vspace{1cm}
\begin{center}
	\begin{tabular}{p{9cm} l}
		& Выполнил студент $610$ группы\\
		& Пехтерев Станислав Игоревич\\
		& $\phantom{C_n^k=C_n^{n-k}}$\\
		& $\underline{\phantom{\int\limits_a^bf(x)dx=F(b)-F(a)}}$\\
		& подпись студента\\
		& $\phantom{\int\limits_f(z)dz=0}$\\
		& Научный руководитель:\\
		& доктор физико-математических наук \\
		& Васенин Валерий Александрович\\
		& $\phantom{C_n^k=C_n^{n-k}}$\\
		& $\underline{\phantom{\int\limits_a^bf(x)dx=F(b)-F(a)}}$\\
		& подпись научного руководителя\\
	\end{tabular}
\end{center}
\vspace{1cm}
\begin{center}
	Москва\\
	$2022$
\end{center}

\newpage
\pagestyle{plain}
\tableofcontents{}
\newpage	
	
 \section*{Тема}



\section{Введение}


\newpage
\section{Постановка задачи}

Для начала поставим общую задачу оптимизации транспортного потока.

%{Пусть задан граф $G = (V, E)$, описывающий некоторую \textit{дорожную сеть}. Предположим, что имеется $n$ участников движения по этому графу. Каждый участник $i$ имеет точки отправления $A_i \in V$ и точки прибытия $B_i \in V$. Пусть множество $P_i$ есть множество всех простых путей из $A_i$ в $B_i$. Пусть декартово произведение ${P = \prod \limits_{i = 1} ^ n P_i}$ есть множество всех возможных комбинаций путей участников. Элементы этого множества назовем \textit{комбинацией путей}. Пусть известно, что при комбинации путей участников ${\textbf{p} \in P}$ $i$-ый участник затрачивает $T_i(\textbf{p})$ времени на передвижение. }

\subsection{Общая постановка задачи}

Пусть задан ориентированный граф $G = (V, E)$, описывающий некоторую \textit{дорожную сеть}. Предположим, что имеется $n$ участников движения по этому графу. Каждый участник $i$ имеет точки отправления $A_i \in V$ и точки прибытия $B_i \in V$. Пусть множество $P_i$ есть множество всех простых путей из $A_i$ в $B_i$. Пусть декартово произведение ${P = \prod \limits_{i = 1} ^ n P_i}$ есть множество всех возможных комбинаций путей участников. Элементы этого множества назовем \textit{комбинацией путей}. Пусть известно, что при комбинации путей участников ${\textbf{p} \in P}$ $i$-ый участник затрачивает $T_i(\textbf{p}) \in \mathbb{R}_{\ge 0}$ времени на путь. 
Функции $T_i$ назовем \textit{функциями временных затрат} участника $i$.
\textit{Некооперативным прокладыванием пути в ориентированном графе G} назовем пятерку $F = (n, G, \{A_i\}_{i = 1}^{n}, \{B_i\}_{i = 1}^{n}, \{T_i\}_{i = 1}^{n})$. Некооперативное прокладывание пути предпологает, что каждый участник стремится сократить собственные временные затраты путем выбора пути $\textbf{p}_i$, невзирая на временные затраты других участников. 
Для того чтобы скооперировать участников, введем некоторую функцию $\Phi (\textbf{p}) = \phi (T_1 (\textbf{p}), \ldots, T_n(\textbf{p}))$, определенную на множестве всех возможных комбинацией путей $P$ и отображающая его в множество действительных чисел, посредством которой они могут отслеживать как влияет изменение их пути на общую картину движения. Такую функцию назовем \textit{функцией стоимости}.

Для заданного некооперативного прокладвания пути $F$ и функции стоимости $\Phi$ необходимо найти такую комбинацию путей $\textbf{p}^*$, что функция стоимости минимальна на ней, то есть

\begin{equation}
	\label{eq:target_global_task_T} 
	\Phi (\textbf{p}^*) = \minn\limits_{ \textbf{p} \in P} \Phi (\textbf{p}).
\end{equation}

Комбинацию путей $\textbf{p}^*$ будем называть \textit {оптимальной}, а стоимость  $ \Phi (\textbf{p}^*)$ \textit {оптимальной стоимостью}.

Далее будем считать, что каждый участник имеет одинаковый приоритет в вопросе изменения своих временных затрат, то есть 

\begin{equation*}
	\frac{\partial \phi}{\partial T_i} \equiv 1, \text{ } i = 1, \ldots, n,
\end{equation*}

или, 

\begin{equation*}
\phi(T_1, \ldots, T_n) = \sum\limits_{i = 1}^nT_i.
\end{equation*}

\subsection{Постановка задачи в терминах модели движения}

Сложность численного решения задачи поиска оптимальной комбинации путей во многом зависит от аналитического задания функций $T_i (\textbf{p})$. Далее приведем ряд ограничений на функции $T_i (\textbf{p})$ нашей задачи, которую собираемся исследовать.

Будем считать, что на временные затраты при проезде по пути $\textbf{p}_i$ в первую очередь влияют временные затраты на ребрах, составляющих маршрут $\textbf{p}_i$. Поэтому без ограничения общности считаем, что функции временных затрат есть суммарные временные затраты на каждом ребре этого пути.

$$T_i (\textbf{p}) = \sum \limits_{e \in E} \overline{\tau}_{e, i} (\textbf{p}) = \sum \limits_{e \in \textbf{p}_i} \overline{\tau}_{e, i} (\textbf{p}), $$

где функции $\overline{\tau}_{e, i} (\textbf{p})$ есть временные затраты $i$-ого участника на ребре $e$ при комбинации путей $\textbf{p}$. 

Будем рассматривать те некооперативные прокладывания пути, в которых существует движение каждого участника для каждой комбинации путей $\textbf{p}$, то есть в каждый момент времени $t \in \mathbb{R}$ известно положение участника на пути. Таким образом, будем считать, что для каждой комбинации путей $\textbf{p}$ и времени $t$ известно присутствует ли участник $i$ на ребре $e$, то есть известны функции

$$
\theta_{e, i} (\textbf{p}, t) =
\begin{cases}
	1, & \text{если }  i\text{-ый участник движется по ребру $e$ в момент времени $t$,}  \\
	0, & \text{иначе},
\end{cases}
$$

где $\sum\limits_{e \in E} \theta_{e, i} (\textbf{p}, t)$ принимает значение $1$ пока участник не доедет до своей точки назначения $B_i$, и $0$ после. Также будем считать, что достижение  участником $i$ вершины $B_i$ наступает в момент $T_i(\textbf{p})$, то есть 
	
\begin{equation}
	\label{eq:T_i_by_theta}
	T_i(\textbf{p}) = \sum \limits_{e \in \textbf{p}_i} \int\limits_{0}^{T_i(\textbf{p}) + \Delta t} \theta_{e, i} (\textbf{p}, t) dt, \forall \Delta t > 0.
\end{equation}

Будем считать, что передвижение каждого учаcтника является непрерывным и последовательным относительно графа $G$. Другими словами участник не может резко повлятся и исчезать на несмежных ребрах, а также посещать пройденные ребра. Таким образом считаем, что функции $\theta_{e, i} (\textbf{p}, t)$ являются индикаторами некоторых интервалов $[t_{e, i}^{in} (\textbf{p}), t_{e, i}^{out} (\textbf{p})]$, которые описывают последовательное передвижение:

\begin{equation}
	\label{eq:restr_t}
	\begin{cases}
		t_{e, i}^{in}(\textbf{p}), t_{e, i}^{out}(\textbf{p}) \in \mathbb{R}_+,  & i = 1, \dots, n, \text{ } e \in E, \\
		t_{e, i}^{in}(\textbf{p}) \le t_{e, i}^{out}(\textbf{p}), & i = 1, \dots, n, \text{ } e \in \textbf{p}_i,  \\
		t_{e, i}^{in}(\textbf{p}) = t_{e, i}^{out}(\textbf{p}) = 0, & i = 1, \dots, n, \text{ } e \notin \textbf{p}_i, \\
		t_{e_1, i}^{in} (\textbf{p}) = t_{e_2, i}^{out} (\textbf{p}), & i = 1, \dots, n, \text{ } e_1, e_2 \in \textbf{p}_i, \exists A, B, C \in V: e_1 = (A, B), e_2 = (B, C)\\
		t_{e, i}^{in} (\textbf{p}) = 0, & i = 1, \dots, n, \text{ } e = (A_i, X), X \in V.
	\end{cases}
\end{equation}

Заметим, что выбор таких интервалов пока неоднозначен. Далее считаем, что для каждого ребра $e$, участника $i$ и комбинации путей $\textbf{p}$ каким-то образом выбраны некоторые велечины $t_{e, i}^{in}(\textbf{p}), t_{e, i}^{out}(\textbf{p})$, удовлетворяющие ограничениям \eqref{eq:restr_t}. Тогда функция временных затрат \eqref{eq:T_i_by_theta} $i$-ого участника примет вид

\begin{equation}
	\label{eq:T_i_by_t}
	T_i(\textbf{p}) = \sum \limits_{e \in E} t_{e, i}^{out}(\textbf{p}) - t_{e, i}^{in}(\textbf{p}).
\end{equation}

Функция стоимости в этом случае есть 

\begin{equation}
	\label{eq:target_func}
	\Phi(\textbf{p}) =\sum \limits_{i = 1}^n \sum \limits_{e \in E} t_{e, i}^{out}(\textbf{p}) - t_{e, i}^{in}(\textbf{p}).
\end{equation}

Без ограничения общности считаем, что временные затраты участником $i$ на ребре $e$ ограничены некоторыми положительными константами $\overline{\tau}_{e, i}^{min}, \overline{\tau}_{e, i}^{max}$:

\begin{equation}
	\label{eq:add_restr}
		0 < \overline{\tau}_{e, i}^{min} \le t_{e, i}^{out}(\textbf{p}) - t_{e, i}^{in}(\textbf{p}) \le \overline{\tau}_{e, i}^{max}, e \in \textbf{p}_i,\, i = 1, \dots, n,
\end{equation}

Заметим, что задача оптимизации с целевой функцией \eqref{eq:target_func} и ограничениями \eqref{eq:restr_t}, \eqref{eq:add_restr} ставится в терминах задачи смешанного целочисленного линейного программирования с вещественными переменными $t_{e, i}^{in}, t_{e, i}^{out} \in \mathbb{R}_+$, отвечающими за моменты прохождения $i$-ым участником ребра $e$ и булевыми переменными $I_{e, i} \in \{0, 1\}$, отвечающими за проезд по ребру $e$ участником $i$. Однако в данных ограничениях решение уже имеется --- участник $i$ передвигается по кратчайшему пути в графе $G$ с весами $\overline{\tau}_{e, i}^{min}$. Тревиальность решения связана с тем, что в данной задаче оптимизации отсутвутют влияния участников друг на друга. Для того, чтобы учесть это влияние, для каждого участника $i$ введем микроскопическую характеристику движения $v_i(\textbf{p}, t)$ ~--- положительная, ограниченная функция, описывающую скорость участника. Также будем считать, что для каждого ребра $e \in E$ определена его длина $l_e > 0$

Тогда, имеет место следующее ограничение

\begin{equation}
	\label{eq:velocity_eq_by_theta}
	\int\limits_{0}^{T_i(\textbf{p})} \theta_{e, i} (\textbf{p}, t) v_i(\textbf{p}, t) dt = l_e, e \in \textbf{p}_i, i = 1, \dots, n,
\end{equation}

или,

\begin{equation}
	\label{eq:velocity_eq_by_t}
	\int\limits_{t_{e, i}^{in}(\textbf{p})}^{t_{e, i}^{out}(\textbf{p})} v_i(\textbf{p}, t) dt = l_e, e \in \textbf{p}_i, i = 1, \dots, n,
\end{equation}

Будем говорить, что уравнения \eqref{eq:velocity_eq_by_t} задают \textit{движения участников}, а функции $v_i(\textbf{p}, t)$ назовем моделью движения. Без ограничения общности считаем, что $\overline{\tau}_{e, i}^{min}, \overline{\tau}_{e, i}^{max}$ вычисляются в самом быстром и самом медленном варианте передвижения по ребру $e$ участником $i$, а именно

\begin{equation}
	\label{eq:restr_add_concrete}
	\overline{\tau}_{e, i}^{min} = \frac{l_e}{\max\limits_{\textbf{p} \in P, t \in \mathbb{R}} \left(  v_i(\textbf{p}, t) \right)}, \overline{\tau}_{e, i}^{max} = \frac{l_e}{\min\limits_{\textbf{p} \in P, t \in \mathbb{R}} \left(  v_i(\textbf{p}, t) \right)}
\end{equation}

Заметим, что велечины $t_{e, i}^{in}(\textbf{p}), t_{e, i}^{out}(\textbf{p}) \in \mathbb{R}_+$ - произвольные вещественные велечины, которые удовлетворяют ограничениям \eqref{eq:restr_t}, \eqref{eq:add_restr}, \eqref{eq:velocity_eq_by_t}, \eqref{eq:restr_add_concrete}.

\begin{state}
\label{state:modeling}
Пусть задан ориентированный граф $G$ с положительными длинами $\{l_e\}_{e \in E}$, модель движения $v_i(\textbf{p}, t)$ и для каждого ребра $e$, участника $i$ и комбинации путей $\textbf{p}$ задано множество велечин $t_{e, i}^{in}(\textbf{p}), t_{e, i}^{out}(\textbf{p}) \in \mathbb{R}_+$, для которых выполняются ограничения \eqref{eq:restr_t}, \eqref{eq:add_restr}, \eqref{eq:velocity_eq_by_t}, \eqref{eq:restr_add_concrete}. Тогда $t_{e, i}^{out}(\textbf{p})$ и $t_{e, i}^{in}(\textbf{p})$, $e \in \textbf{p}_i$ есть функции от комбинации путей $\textbf{p} \in P$.
\end{state}
\begin{proof}
Зафиксируем некоторую комбинацию путей $\textbf{p}$. Опишем алгоритм поиска значений $t_{e, i}^{out}(\textbf{p})$ и $t_{e, i}^{in}(\textbf{p})$ и покажем его корректность.

\begin{algorithm}
	\caption{Моделирование движения участников}
	\label{alg:modeling}
	{\bf {Input:}} количество участников $n$, ориентированный граф $G$, комбинация путей $\textbf{p}$ графа $G$\\
	{\bf {Output:}} $t_{e, i}^{out}(\textbf{p})$, $t_{e, i}^{in}(\textbf{p})$, $e \in \textbf{p}_i, i = 1, \ldots, n$\\
	{\bf {Data:}} текущее время $t$, текущее ребро $e_i$ и часть пройденного ребра $x_i$ участника $i$
	\begin{algorithmic}[1]
		\State $t = 0$
		\For{$i = 1, \ldots, n$}
		\State $e_i \gets$ { первое ребро пути $\textbf{p}_i$}
		\State $x_i \gets 0$
		\State $t_{e_i, i}^{in}(\textbf{p}) \gets 0$ 
		\EndFor
		\While{$\exists i: i \text { --- не доехал}$}
		\State $\tau^* \gets \argmin\{ \tau \in \mathbb{R}: \tau > t, \int\limits_{t}^{\tau} v_i(\textbf{p}, t) dt = (1 - x_i) l_{e_i}, i \text{--- не доехал}  \}_{i = 1}^n$
		\For{$i = 1, \ldots, n$ \textbf{and} $i - \text{ не доехал}$ }
		\State $x_i \gets x_i + \frac{1}{l_{e_i}} \int\limits_{t}^{\tau^*} v_i(\textbf{p}, t) dt$
			\If{$x_i = 1$ \textbf{and} $e_i$ - не последнее ребро пути $\textbf{p}_i$ }
				\State $x_i \gets 0$
				\State $t_{e_i, i}^{out}(\textbf{p}) \gets \tau^*$ 
				\State $e_i \gets$ следующее ребро за $e_i$ в пути $\textbf{p}_i$
				\State $t_{e_i, i}^{in}(\textbf{p}) \gets \tau^*$ 
			\EndIf
		\EndFor
		\State $t \gets \tau^*$
		\EndWhile
	\end{algorithmic}
\end{algorithm}

Описанный алгоритм называется \textit{моделированием движения}.

\textit{Корректность}. Корректность шага 8 алгоритма и достижимость шага 11 следует из того, что функция скорости ограничена снизу (см. ограничения \eqref{eq:add_restr}, \eqref{eq:restr_add_concrete}). Алгоритм сойдется, поскольку пути $\textbf{p}_i$ конечны.

\end{proof}
 Используя это утверждение, задача \eqref{eq:target_global_task_T} с учетом введенных ограничений \eqref{eq:restr_t}, \eqref{eq:T_i_by_t}, \eqref{eq:add_restr}, \eqref{eq:velocity_eq_by_t}, \eqref{eq:restr_add_concrete} можно ввести следующее понятие:

\textit{Некооперативным передвижением} по графу $G$ c положительными длинами $\{l_e\}_{e \in E}$ в модели движения $v_i(\textbf{p}, t)$ назовем такое некооперативное прокладывание пути $F = (n, G, \{A_i\}_{i = 1}^{n}, \{B_i\}_{i = 1}^{n}, \{\sum_{e \in \textbf{p}_i} t_{e, i}^{out}(\textbf{p}) - t_{e, i}^{in}(\textbf{p})\}_{i = 1}^{n})$, где функции $t_{e, i}^{in}(\textbf{p}), t_{e, i}^{out}(\textbf{p})$ получены путем моделирования движения с моделью движения $v_i(\textbf{p}, t)$ и длинами ребер $\{l_e\}_{e \in E}$. Далее считаем, что длины ребер 
Значит, постановка задачи в терминах модели движение следующая:

Пусть задано некооперативное передвижениепо графу $G$ c положительными длинами $\{l_e\}_{e \in E}$ в модели движения $v_i(\textbf{p}, t)$.
Требуется найти такую комбинацию путей $\textbf{p}$, что функция 

\begin{equation}
\label{eq:target_task_end}
\Phi(\textbf{p}) =\sum \limits_{i = 1}^n \sum \limits_{e \in E} t_{e, i}^{out}(\textbf{p}) - t_{e, i}^{in}(\textbf{p})
\end{equation}

минимальна.

Оказывается, что для любого некооперативного прокладыванием пути при любых положительных длинах $\{l_e\}_{e \in E}$, существет эквивалентное ему кооперативное передвижением в графе с этими длинами в некоторой модели движения $v_i(\textbf{p}, t)$. Другими словами любое некооперативное прокладывание пути можно как-то промоделировать.

\newpage
\section{Модели движения}

\begin{state}
	\label{state:eqv}
	Пусть задано некоторое некооперативное прокладыванием пути $F$ и положительные длины ребер  $\{l_e\}_{e \in E}$.Тогда можно задать такую модель движения $v_i(\textbf{p}, t)$, что затраченное время на передвижение $i$-ым участником при комбинации путей $\textbf{p}$ совпадает с его временными затратами, то есть выполняется \eqref{eq:T_i_by_t}.
	
\end{state}

\begin{proof}
Рассмотрим модель движения с постоянными скоростями  
$$v_i(\textbf{p}, t) = \overline{v}_i(\textbf{p}) = \frac{T_i (\textbf{p})}{\sum \limits_{e \in \textbf{p}_i} l_e}.$$

Промоделировав движение с такими скоростями получим \eqref{eq:T_i_by_t}.
\end{proof}

Таким образом можно сказать, что без ограничения общности считать, что каждый выбор комбинации путей $\textbf{p}$ можно промоделировать.

Очевидно, что решение задачи перебором не является практичным --- оно сводится к перебору всех комбинаций путей $\textbf{p} \in P$. Так, например, количество таких комбинаций в полном графе есть $2^{n (|V| - 1)}$, перебрать которые в условиях реальных данных вычислительно сложно.
Однако, в случае, когда условие \eqref{eq:velocity_eq_by_t} можно описать в терминах задачи удовлетворения ограничений, задача оптимизации \eqref{eq:target_task_end} может быть описана в терминах смешанного целочисленного линейного программирование и, как следствие, может быть решена стандартным решателем. Оказывается можно выделить целый класс таких моделей движения, для которых это возможно.

\subsection{Макроскопические модели}

Предположим, что скорость участника зависит от некоторой общей для участников велечины. Например, от функции загруженности ребра

$$ n_{e}(\textbf{p}, t) = \sum\limits_{i = 1}^n\theta_{e, i}(\textbf{p}, t),$$
значение которой в момент времени $t$ соответствует количеству участников на ребре $e$ в этот момент при комбинации путей $\textbf{p}$. Предположим скорость участника зависит только от загруженности ребра, на котором он находится:

\begin{equation}
	\label{eq:velocity_eq_macro}
	 v_i(\textbf{p}, t) = \sum \limits _{e \in E} \theta_{e, i} (\textbf{p}, t) v (n_e (\textbf{p}, t)),  i = 1, \dots, n
\end{equation}

Такую модель движения в дальнейшем будем называть \textit{макроскопической}.
Например, естествено расмотреть модель $ v (n_e (\textbf{p}, t)) = \frac{v_{max}}{n_e (\textbf{p}, t)}$. В общем случае такая модель задается последовательностью значений  $\{v(k)\}_{k = 1}^n$.

\begin{lemma}
	\label{lemma:lt}
	Пусть даны вещественные переменные $a$, $b$ целочисленного программирования и известно, что существует константа $M > 0$ : $|a| < M$, $|b| < M$. Тогда можно добавить новую целочисленную переменную $\textbf{1} (\{a < b\}) \in \{0, 1\}$ такую, что
	
	\begin{equation*}
		\textbf{1} (\{a < b\}) = 
		\begin{cases}
			1,\, a < b,
			\\
			0,\, a \ge b.
		\end{cases}
	\end{equation*}

\end{lemma}

\begin{proof}
	Добавим в нашу задачу два неравенства:
	
	$$ 2M (\textbf{1} (\{a < b\}) - 1) < b - a \le 2M\textbf{1} (\{a < b\}) $$
	
	Очевидная проверка показывает, что неравенство выполняется для любых $a, b$.
	
	
\end{proof}

\begin{state}
	
	\label{state:lin_prog}
	
	Пусть модель движения $ v_i(\textbf{p}, t)$ макроскопическая. Тогда задача \eqref{eq:target_task_end} есть задача смешанного целочисленного линейного программирования.
\end{state}

\begin{proof}
	Докажем для случая $n = 2$. Для случаев $n \ge 2$ доказательство аналогичное.
	
	Пусть имеется задача смешанного целочисленного линейного программирования \eqref{eq:restr_t} с переменными $t_{e, i}^{in}, t_{e, i}^{out}, I_{e, i}, e \in E, i = 1, 2$. Преобразуем условие \eqref{eq:velocity_eq_by_t} к каноническому виду задачи удовлетворения ограничений. Для удобства обозначим обоих участников индексами $i, j \in \{1, 2\}$.
	
	$$\int\limits_{0}^{\infty} \theta_{e, i} (\textbf{p}, t) v_i(\textbf{p}, t)dt = \int\limits_{0}^{\infty} \theta_{e, i} (\textbf{p}, t) \sum \limits _{e^1 \in E} \theta_{e^1, i} (\textbf{p}, t) v (n_{e^1} (\textbf{p}, t)) dt = $$
	
	$$\int\limits_{0}^{\infty} \theta_{e, i} (\textbf{p}, t)  v (n_{e} (\textbf{p}, t)) dt = 
	  \int\limits_{ \substack{n_{e} (\textbf{p}, t) = 1}} \theta_{e, i} (\textbf{p}, t)  v (n_{e} (\textbf{p}, t)) dt +
	  \int\limits_{ \substack{n_{e} (\textbf{p}, t) = 2}} \theta_{e, i} (\textbf{p}, t)  v (n_{e} (\textbf{p}, t)) dt = $$
	
	$$\int\limits_{ \substack{\theta_{e, j} (\textbf{p}, t) = 0}} \theta_{e, i} (\textbf{p}, t)  v (1) dt +
	  \int\limits_{ \substack{\theta_{e, j} (\textbf{p}, t) = 1}} \theta_{e, i} (\textbf{p}, t)  v (2) dt = $$
	  
    $$\int\limits_{0}^{\infty} \theta_{e, i} (\textbf{p}, t)  v (1) dt - 
      \int\limits_{ \substack{\theta_{e, j} (\textbf{p}, t) = 1}} \theta_{e, i} (\textbf{p}, t)  v (1) dt +
	  \int\limits_{ \substack{\theta_{e, j} (\textbf{p}, t) = 1}} \theta_{e, i} (\textbf{p}, t)  v (2) dt = $$
	  
	$$v (1) \int\limits_{0}^{\infty} \theta_{e, i} (\textbf{p}, t) dt +
	  (v (2) - v(1)) \int\limits_{0}^{\infty} \theta_{e, i} (\textbf{p}, t) \theta_{e, j} (\textbf{p}, t) dt = $$
	  
    $$v (1) \overline{\tau}_{e, i} (\textbf{p}) +
    (v (2) - v(1)) \int\limits_{0}^{\infty} \theta_{e, i} (\textbf{p}, t) \theta_{e, j} (\textbf{p}, t) dt = l_e, e \in \textbf{p}_i$$
	  
	Неизвестный интеграл - время совместного проезда участников на ребре $e$.
	  
	В переменных задачи смешанного целочисленного программирования получим:
	
	$$v(1) (t_{e, i}^{out} - t_{e, i}^{in}) + (v(2) - v(1)) (t_{e, ij}^{out} - t_{e, ij}^{in}) = l_e I_{e, i},$$
	
	где новые перемнные $t_{e, ij}^{in}$, $t_{e, ij}^{out}$ отвечают за начало и конец совместного проезда участников. Другими словами $[t_{e, ij}^{in}, t_{e, ij}^{out}] = [t_{e, i}^{in}, t_{e, i}^{out}] \cap [t_{e, j}^{in}, t_{e, j}^{out}]$. Просуммировав по всем ребрам $e \in E$, получим
	
	$$v(1) \sum \limits _{e \in E} (t_{e, i}^{out} - t_{e, i}^{in}) = \sum \limits _{e \in E} l_e I_{e, i} - (v(2) - v(1)) \sum \limits _{e \in E} (t_{e, ij}^{out} - t_{e, ij}^{in})$$
	
	Заметим, что левая часть есть временные затраты участника $i$ с коэффициентом $v(1)$, поэтому задачу оптимизации можно переписать в виде
	$$ \frac{1}{v (1)} \sum\limits_{i = 1}^n \sum \limits _{e \in E} l_e I_{e, i} + \frac{v(1) - v(2)}{v (1)}  \sum\limits_{i = 1}^n \sum \limits _{e \in E} (t_{e, ij}^{out} - t_{e, ij}^{in}) \rightarrow \minn $$
	
	Для завершения доказательства необходимо показать, что переменные $t_{e, ij}^{in}, t_{e, ij}^{out}$ описываются линейными ограничениями. Обозначим $\Delta t = t_{e, ij}^{out} - t_{e, ij}^{in}, \Delta t_1 =  t_{e, i}^{out} - t_{e, i}^{in}, \Delta t_2 =  t_{e, j}^{out} - t_{e, j}^{in}, \Delta t_3 =  t_{e, i}^{out} - t_{e, j}^{in}, \Delta t_4 =  t_{e, j}^{out} - t_{e, i}^{in}$
	
	Используя лемму \ref{lemma:lt}, при $M = \max\limits_{e \in E, k = i, j} \overline{\tau}_{e, k}^{max}$, добавим в задачу новые переменные $\textbf{1} (\{ \Delta t_k > \Delta t_l\}), k \ne l, k, l \in {1, 2, 3, 4}$. Рассмотрим велечину $T_{max} = |E|M$. Добавим в случае $v(1) \ge v(2)$ нашу задачу следующие неравенства: 
	
	$$\Delta t \ge 0, $$
	$$\Delta t \ge \Delta t_k - T_{max} \sum \limits_{l \ne k} {\textbf{1} (\{ \Delta t_k > \Delta t_l\})}, k = 1, 2, 3, 4.$$
	
	В случае $v(1) < v(2)$ добавим те же ограничения с другим знаком неравенства. Тогда с учетом оптимизации переменная $\Delta t$ есть длина отрезка $[t_{e, ij}^{in}, t_{e, ij}^{out}]$. 
	
\end{proof}

\begin{corollary}
	\label{corollary:rel}

	Пусть модель движения $ v_i(\textbf{p}, t) = \sum \limits _{e \in E} \theta_{e, i} (\textbf{p}, t) v (n_e (\textbf{p}, t))$ макроскопическая и последовательность $v(n) > 0, \forall n \in \mathbb{Z}_+$ убывает. Предположим, что оптимальное время движения в модели c постоянными скоростями $v(1)$ есть $\widetilde{T}$. Тогда имеет место

	$$ \widetilde{T} \le T \le \frac {v(1)}{v(n)} \widetilde{T}.$$
	
\end{corollary}

\begin{proof}
	Докажем каждое неравенство в отдельности
	
	1. В модели, где все участники едут с постоянными скоростям движение происходит по кратчайшим путям. Тогда временные затраты есть $\widetilde{T} = \frac{1}{v(1)} \sum \limits _{i = 1} ^ n \sum\limits_{e \in p_i} l_e$, где $p_i$ - кратчайшие пути.
	На тех же путях задается самый худший случай макроскопической модели - все едут с минимальной скоростью, то есть $ T = \frac{1}{v(n)} \sum \limits _{i = 1} ^ n \sum\limits_{e \in p_i} l_e$. Тогда получим
	
	$$T \le  \frac{1}{v(n)} \sum \limits _{i = 1} ^ n \sum\limits_{e \in p_i} l_e = \frac {v(1)}{v(n)} \widetilde{T}.$$
	
	2. Проделывая аналогичне выкладки что и в доказательстве \ref{state:lin_prog}, можно получить, что функция оптимизация есть 
	
	$$ \frac{1}{v (1)} \sum\limits_{i = 1}^n \sum \limits _{e \in E} l_e I_{e, i} +  \sum\limits_{k = 2}^{n} \frac{v(1) - v(k)}{v (1)}  \sum\limits_{i = 1}^n \sum \limits _{e \in E} \sum\limits _{\substack{ s_k \in 2^n \\ |s_k| = k}}  \Delta t_{e, s_k} \rightarrow \minn ,$$
	
	где переменные $\Delta t_{e, s_k}$ отвечают за время совместного движения участников (и только их) $s_k$ по ребру $e$.
	
	Тогда получим
		$$T \ge 
		  \minn \left(  \frac{1}{v (1)} \sum\limits_{i = 1}^n \sum \limits _{e \in E} l_e I_{e, i} \right) 
		+ \minn \left(  \sum\limits_{k = 2}^{n} \frac{v(1) - v(k)}{v (1)}  \sum\limits_{i = 1}^n \sum \limits _{e \in E} \sum\limits _{\substack{ s_k \in 2^n \\ |s_k| = k}}  \Delta t_{e, s_k} \right) \ge \widetilde{T}.$$

\end{proof}

Таким образом, мы получили класс моделей движения, для которых задача оптимизации транспортного потока может быть поставлена в терминах смешанного целочисленного линейного программирования. Однако такой класс моделей движения плохо описывает реальное движение участников. Так, например, модель не учитывает расстояние между участниками и их порядок на ребре.

\subsection{Микроскопические модели}
\textit {Микроскопическими} называются модели движения, которые не являются макроскопическими, то есть не представимы в виде \eqref{eq:velocity_eq_macro}. В таких моделях явно исследуется движение каждого автомобиля.
Выбор такой модели позволяет теоретически достичь более точного описания движения автомобилей по сравнению с макроскопической моделью, однако этот подход требует больших вычислительных ресурсов при практических применениях.

Для простоты рассмотрим однополосное бесконечное движение. Пусть ${x_i(t) \in [0, +\infty)}$~--- координаты на полосе участника $i$. Предположим, что скорость участника ограничена некоторой общей велечиной $v_{max}$. Пусть в момент времени ${t = 0}$ выполняется $x_1(0) \le x_2(0) \le \dots \le x_n(0)$.

\subsubsection*{Модель пропорциональной скорости}
Рассмотрим пример, когда скорость машины пропорциональна расстоянию до следующей машины.
Положим $d_{i} (t) = x_{i + 1} (t) - x_{i} (t), i = 1, \dots, n - 1$.
Без ограничения общности считаем, что  $d_{i} (0) < D$, где $D$ - характерное расстояние взаимодействия участников. Иначе рассмотрим подпоследовательности участников, для которых выполняется это условие.

Пусть модель движения есть
\begin{equation}
	\label{eq:micro}
	v_i(t)=
	\begin{cases}
		v_{max}, & i = n,
		\\
		v_{max} \frac{d_i(t)}{D} ,& i \ne n.
	\end{cases}
\end{equation}

Для поиска функций $x_i(t)$ достаточно рассмотреть систему дифференциальных уравнений

$$ \dot{d_i} (t) = v_{i + 1} (t) - v_i (t).$$

Решением такой системы является

$$d_{n - k} (\tau) = \sum \limits_{l = 0} ^ {k - 1} \left(\frac{d_{n - k + l} (0) - D}{l!} \tau^l e ^ {-\tau}\right) + D ,$$
где ${\tau = \frac{v_{max}}{D}t}$. Модель обладает тем свойством, что порядок участников постоянен и участники не покидают зону взаимодействия $D$. 

Данная модель хорошо описывает реальное движение участников, однако ее практическое применение вызывает сложности, поскольку решение уравнения время, вычисляемое на шаге 2 процесса моделирования движения может быть найдено только приближенно.

\subsubsection*{Модель снижения скорости}

Предположим, что существует некоторая велечина $c_n$, которая отвечает за последовательное снижение скорости участников относительно их порядка:

$$v_{n - k} = v_{max} - c_n k, \quad k = 0, \dots, n - 1$$

Велечину $c_n$ выберем из соображений, что $v_0 = \frac{v_{max}}{n}$. Тогда $c_n = \frac{v_{max}}{n}$. Если смоделировать данное движение на графе, то функция скоростей будут кусочно постоянными. Это связано с тем, при смене ребра некоторым участником меняется порядок и велечина $n_e(\textbf{p}, t)$. Поэтому она не лучшим образом описывает реальное движение, однако проста в использовании.

Пока для микроскопических моделей нет очевидного подхода к решению. Однако для любой модели движения можно описать алгоритмы оптимизации, которые сходится к <<локальному минимуму>>. Рассмотрим такие алгоритмы в следующем разделе.

\newpage
\section{Равновесие транспортных потоков}
\label{sec:rovn}
В этом разделе мы исследуем задачу поиска равновесия транспортных потоков как возможность поиска оптимального транспортного потока.

\subsection{Некооперативное и кооперативное равновесие}

\textit{Некооперативной игрой в нормальной форме} назовем тройку $\Gamma = (n, \{S_i\}_{i = 1}^n, \{H_i\}_{i = 1}^n)$, где $n \in \mathbb{N}$ - количество участников игры, $S_i$ - множество стратегий участника $i \in {1, \dots, n}$, $H_i$ - функция выйгрыша участника $i$, определенная на множестве ситуаций $S = \prod\limits_{i = 1}^n S_i$ и отображающая его в множество действительных чисел.

\textit{Равновесием Нэша} некооперативной игры в нормальной форме $\Gamma = (n, \{S_i\}_{i = 1}^n, \{H_i\}_{i = 1}^n)$ назовем такую стратегию $\textbf{s}^* \in S$, если изменение своей стратегии с $\textbf{s}_i^*$ на любую $s \in S_i$ не выгодно ни одному игроку $i$, то есть

$$H_i(\textbf{s}^*) \ge H_i(\left(\textbf{s}^*_1, \ldots, \textbf{s}^*_{i - 1}, s, \textbf{s}^*_{i + 1}, \ldots, \textbf{s}^*_{n} \right)), \forall s \in S_i, \, i = 1, \dots, n. $$ 

Заметим, что в общем случае ничего нельзя сказать о существовании и единственности равновесия некооперативной игры.

Введем понятие некооперативного и кооперативного равновесия, которое является классическим определением равновесия в терминах некооперативного прокладывания пути, где выйгрыш заключается в сэкономленном времени передвижения и стоимости соответственно.

\textit{Некооперативным равновесием} некооперативного прокладывания пути по графу $F$ назовем комбинацию путей $\widehat{\textbf{p}} \in P$, которая является равновесием Нэша некооперативной игры $\widehat{\Gamma} = (n, \{P_i\}_{i = 1}^n, \{-T_i\}_{i = 1}^n)$. Множество всех некооперативных равновесий обозначим $\widehat{P}$.

\textit{Кооперативным равновесием} некооперативного прокладывания пути по графу $F$ и функции стоимости $\Phi (\textbf{p})$ назовем комбинацию путей $\widetilde{\textbf{p}} \in P$, которая является равновесием Нэша некооперативной игры $\widetilde{\Gamma} = (n, \{P_i\}_{i = 1}^n, \{-\Phi\}_{i = 1}^n)$. Множество всех кооперативных равновесий обозначим $\widetilde{P}$.

Заметим, что определения некооперативного прокладывания пути и некооперативной игры эквивалентны.  
Таким образом, любой пример игры, где равновесия Нэша не существует, можно использовать как пример некооперативного передвижения по графу, где нет кооперативного равновесия.
Однако для кооперативного равновесия верно обратное:

\begin{state}
Множество кооперативных равновесий $\widetilde{P}$ не пусто, причем
оптимальная комбинация путей является таким равновесием, то есть $\textbf{p}^* \in \widetilde{P}$.
\end{state}

\begin{proof}
	Очевидно, что для поскольку для любого $\textbf{p} \in P$ 
	$$\Phi (\textbf{p}^*) \le \Phi (\textbf{p}),$$
	поэтому это верно и для комбинаций путей $\textbf{p} = \left(\textbf{p}^*_1, \ldots, \textbf{p}^*_{i - 1}, p, \textbf{p}^*_{i + 1}, \ldots, \textbf{p}^*_{n} \right), \text{ } p \in P_i, \text{ } i = 1, \ldots, n$.
\end{proof}

В некотором смысле кооперативное равновесие можно назвать <<локальным минимум>> функции $\Phi$.

\subsection{Поиск кооперативного равновесия}
Рассмотрим ряд алгоритмов, позволяющих получить некоторое кооперативное равновесие.

Общим свойством всех этих алгоритмов является предположение о том, что существует некоторый алгоритм $A$, позволяющий решить задачу оптимизации некоторой функции стоимости $\Phi_i (\textbf{p}) = \phi_i(T_1(\textbf{p}), \ldots, T_n(\textbf{p}))$ путем выбора пути $\textbf{p}_i$. В работе Л.\,Е.~Разумовой~\cite{Luba} представлен один из таких алгоритмов, позволяющий получить оптимальный путь $\textbf{p}_i$ за полиномиальное относительно входных данных время, при условии, что функция $\Phi_i (\textbf{p})$ удовлетворяет неравенству

\begin{align*}
	\Phi_i (\textbf{p}_1, \ldots, \textbf{p}_{i - 1}, pe, \textbf{p}_{i + 1}, \ldots, \textbf{p}_n) + 
	\Phi_i (\textbf{p}_1, \ldots, \textbf{p}_{i - 1}, p, \textbf{p}_{i + 1}, \ldots, \textbf{p}_n) \le \\ \le
	\Phi_i (\textbf{p}_1, \ldots, \textbf{p}_{i - 1}, qe, \textbf{p}_{i + 1}, \ldots, \textbf{p}_n) +
	\Phi_i (\textbf{p}_1, \ldots, \textbf{p}_{i - 1}, q, \textbf{p}_{i + 1}, \ldots, \textbf{p}_n),
\end{align*}

где $p, q$ --- два пути к некоторой вершине $B \in V$, ребро $e$ выходит из этой вершины и путь $p$ оптимальнее чем $q$ относительно стоимости $\Phi_i$:

$$\Phi_i (\textbf{p}_1, \ldots, \textbf{p}_{i - 1}, p, \textbf{p}_{i + 1}, \ldots, \textbf{p}_n) \le
  \Phi_i (\textbf{p}_1, \ldots, \textbf{p}_{i - 1}, q, \textbf{p}_{i + 1}, \ldots, \textbf{p}_n)$$

\if 0
Заметим, что такое некооперативное равновесие не всегда существует. Рассмотрим следующее некооперативное передвижение $F = (2, G_1, (A, B), (D, D), \{T\}_{i = 1}^{n})$ (см. рис \ref{fig:nocoop}) с некоторой нетривиальной макроскопической моделью движения $(v(1), v(2))$, такой что $v(2) < v(1)$.

\begin{figure}[hpt]
	\includegraphics[scale = 0.3]{imgs/graph_nocoop.png}
	\centering
	\caption{Пример графа $G_1$ без некооперативного равновесия}
	\label{fig:nocoop}
\end{figure}

Будем считать, что для длин ребер выполнены следующие неравенства:

$$ l_{e_2} < l_{e_3} < l_{e_1}, \text{ } l_{e_4} > l_{e_3} - l_{e_2}, \text{ } l_{e_4} > l_{e_1} - l_{e_3}. $$

Данные ограничения необходимы для появления следующих ситуация при выборе пути первым участником (см. рис \ref{fig:nocoop_str}):

\begin{figure}[hpt]
	\includegraphics[scale = 0.3]{imgs/no_nocoop_rovn.png}
	\centering
	\caption{Выгодная стратегия участника 1 (слева) и выгодная стратегия участника 2 (справа).}
	\label{fig:nocoop_str}
\end{figure}

При выборе участником 1 пути $p_1 = Ae_2Ce_4D$ 

Для такого графа считаем, что выполнены следующие неравенства.
$$l2 < l3 < l1,$$
$$l4 > $$
\fi

\newpage
\section{Результаты}

\newpage
\section{Заключение}

    \newpage
\begin{thebibliography}{0}
	
	\addcontentsline{toc}{section}{Литература}

	\bibitem{Luba} \textit{Л.\,Е.~Разумова, С.\,А.~Афонин},  ``Построение оптимального маршрута при заданной модели движения других участников движения''.
	
	\bibitem{Gluts} \textit{А.\,А.~Глуцюк}, ``О двумерных полиномиально интегрируемых бильярдах на поверхностях постоянной кривизны'', Доклады Академии наук, \textbf{481}, 6, 2018, 594--598
	
	\bibitem{MirByal} \textit{M.~Bialy, A.\,E.~Mironov}, ``Angular billiard and algebraic Birkhoff conjecture'', Adv. Math., \textbf{313}, 2017, 102--126
	
\end{thebibliography} 

\end{document}