\documentclass[12pt, a4paper]{article}
\usepackage[utf8]{inputenc}
\usepackage[russian]{babel}
\usepackage[T2A]{fontenc}
\usepackage{amsfonts}
\usepackage{amsmath}
\usepackage{indentfirst}
\DeclareMathOperator*{\minn}{min}

\usepackage[left=2cm,right=1.5cm,top=2cm,bottom=2cm]{geometry}
\linespread{1.25}

\usepackage{graphicx}
\graphicspath{{pictures/}}
\DeclareGraphicsExtensions{.pdf,.png,.jpg}

\begin{document}
 \section*{Тема}
Оптимизация транспортного потока при заданных пунктах отправления и назначения всех участников движения


\section{Введение}


\newpage
\section{Постановка задачи}

Пусть задан граф $G = (V, E)$, описывающий некоторую \textit{дорожную сеть}. Предположим, что имеется $n$ участников движения по этому графу. Каждый участник $i$ имеет точки отправления $A_i \in V$ и точки прибытия $B_i \in V$. Пусть множество $P_i$ - есть множество всех простых путей из $A_i$ в $B_i$. Пусть декартово произведение $P = \prod \limits_{i = 1} ^ n P_i$ есть множество всех возможных комбинаций путей участников. Элементы этого множества назовем \textit{комбинацией путей}. Пусть известно, что при комбинации путей участников $\textbf{p} \in P$ $i$-ый участник затрачивает $T_i(\textbf{p})$ времени на передвижение. Пару $(P, \{T_i\}_{i = 1} ^ n)$ назовем \textit{некооперативным передвижением} на графе $G$. Функции $T_i: P \rightarrow R_+$ назовем \textit{функцией временных затрат}.

Необходимо найти такую комбинацию путей участников $\textbf{p}^*$, что суммарные временные затраты на передвижение - минимальны $$\sum\limits_{i = 1}^n T_i (\textbf{p}^*) = \minn\limits_{ \textbf{p} \in P} \sum\limits_{i = 1}^n T_i (\textbf{p})$$

Комбинацию путей $\textbf{p}^*$ будем называть \textit {оптимальной}, а суммарные временные затраты  $\sum\limits_{i = 1}^n T_i (\textbf{p}^*)$ \textit {оптимальным временем передвижения участников}

\newpage
\section{Поиск функции временных затрат}

Сложность численного решения задачи поиска оптимальной комбинации путей во многом зависит от аналитического задания функций
$T_i (\textbf{p})$. Интуитивно вполне очевидно, что на временные затраты при проезде по пути $\textbf{p}_i$ в первую очередь влияют временные затраты на ребрах, составляющих маршрут $\textbf{p}_i$. Поэтому без ограничения общности считаем, что функции временных затрат есть суммарные временные затраты на каждом ребре этого пути

$$T_i (\textbf{p}) = \sum \limits_{e \in \textbf{p}_i} \overline{\tau}_e (\textbf{p}), $$
где функции $\overline{\tau}_e (\textbf{p})$ есть временные затраты на ребре $e$ при комбинации путей $\textbf{p}$. Поскольку подразумевается, что передвижение участников происходит непрерывно во времени, то, можно считать, что временные затраты на ребре $e$ есть усредненные временные затраты в течении времени движения

$$ \overline{\tau}_e (\textbf{p}) = \int_{0}^{\infty} \tau_e (\textbf{p}, t) dt,$$
где функции  $\tau_e (\textbf{p}, t)$ представляют из себя затраченное время на передвижение по ребру $e$ в момент времени $t$ при комбинации путей $\textbf{p}$. Таким образом, без ограничения общности считаем, что

$$T_i (\textbf{p}) = \sum \limits_{e \in \textbf{p}_i} \int_{0}^{\infty} \tau_e (\textbf{p}, t) dt $$

Далее для простоты изложения будем опускать зависимость функций от выбранной комбинации $\textbf{p}$.

Предположим, что у каждого участника движения имеется микроскопическая характеристика скорости движения $v_i(t)$, которая ограничена некоторой константой $v_{max}$ - максимальной скоростью передвижения. В случае постоянных скоростей интуитивно очевидно, что вклад каждого участника, проехавшего по ребру $e$ есть $\frac{l_e}{v_i}$, где $l_e$ - длина ребра $e$. Обобщим это предположение на случай непостоянных скоростей:

$$  \tau_e (t) = \sum \limits_{j = 1}^n \theta_{e, j} (t) \frac{l_e}{v_j(t)}, $$
где 

$$
\theta_{e, j} (t) =
\begin{cases}
	1, & \text{если}  \text{ j-ый участник движется по ребру $e$ в момент времени $t$}  \\
	0, & \text{иначе}
\end{cases}
$$

Таким образом, введя микроскопическую характеристику скорости движения $v_i(t)$ получим:

$$T_i (\textbf{p}) = \sum \limits_{e \in \textbf{p}_i} l_e \sum \limits_{j = 1}^n \int_{0}^{\infty} \theta_{e, j} (t) \frac{1}{v_j(t)} dt $$


\subsection*{Пример 1}

Предположим, что задана некоторая зависимость скорости участников $v_j(t)$ от загруженности $n_e(t) = \sum\limits_{j = 1}^n\theta_{e, j} (t)$ на текущем ребре (ребре $e$, удовлетворяющем $\theta_{e, j} (t) = 1$).

$$v_j(t) = \sum \limits_{e \in \textbf{p}_i} v (n_e (t)) \theta_{e, j} (t)$$

В этом случае врвменные затраты имеют вид

$$T_i (\textbf{p}) = \sum \limits_{e \in \textbf{p}_i} l_e \int_{0}^{\infty} n_e (t) \frac{1}{v(n_e(t))} dt $$

Такую модель скорости будем в дальнейшем называть \textit{макроскопической моделью}.

Простейшим примером такой модели является модель постоянной суммарной скорости 

$$v (n_e (t)) = \frac{v_{max}}{n_e (t)},$$
или
$$T_i (\textbf{p}) = v_{max} \sum \limits_{e \in \textbf{p}_i} l_e \int_{0}^{\infty} n_e^2 (t) dt $$


\end{document}