\documentclass[12pt, a4paper]{article}
\usepackage[utf8]{inputenc}
\usepackage[russian]{babel}
\usepackage[T2A]{fontenc}
\usepackage{amsfonts}
\usepackage{amsmath}
\usepackage{indentfirst}
\usepackage{amsthm}
\DeclareMathOperator*{\minn}{min}
\newtheorem{theorem}{Theorem}[section]
\newtheorem{state}{Утверждение}[section]
\newtheorem{lemma}{Лемма}[section]
\newtheorem{corollary}{Следствие}[section]


\usepackage[left=2cm,right=1.5cm,top=2cm,bottom=2cm]{geometry}
\linespread{1.25}

\usepackage{graphicx}
\graphicspath{{pictures/}}
\DeclareGraphicsExtensions{.pdf,.png,.jpg}

\begin{document}
	
\begin{center}
	ФЕДЕРАЛЬНОЕ ГОСУДАРСТВЕННОЕ БЮДЖЕТНОЕ ОБРАЗОВАТЕЛЬНОЕ\\
	УЧРЕЖДЕНИЕ ВЫСШЕГО ОБРАЗОВАНИЯ\\
	<<МОСКОВСКИЙ ГОСУДАРСТВЕННЫЙ УНИВЕРСИТЕТ\\
	имени М.\,В.~ЛОМОНОСОВА>>
\end{center}
\vspace{4pt}
\begin{center}
	МЕХАНИКО-МАТЕМАТИЧЕСКИЙ ФАКУЛЬТЕТ
\end{center}
\vspace{4pt}
\begin{center}
	КАФЕДРА ВЫЧИСЛИТЕЛЬНОЙ МАТЕМАТИКИ
\end{center}
\vspace{1cm}
\begin{center}
	ВЫПУСКНАЯ КВАЛИФИКАЦИОННАЯ РАБОТА\\
	специалиста
\end{center}

\begin{center}
	\textbf{ОПТИМИЗАЦИЯ ТРАНСПОРТНОГО ПОТОКА \\
		    ПРИ ЗАДАННЫХ ПУНКТАХ ОТПРАВЛЕНИЯ И НАЗНАЧЕНИЯ \\
		    ВСЕХ УЧАСТНИКОВ ДВИЖЕНИЯ}
\end{center}
\vspace{1cm}
\begin{center}
	\begin{tabular}{p{9cm} l}
		& Выполнил студент $610$ группы\\
		& Пехтерев Станислав Игоревич\\
		& $\phantom{C_n^k=C_n^{n-k}}$\\
		& $\underline{\phantom{\int\limits_a^bf(x)dx=F(b)-F(a)}}$\\
		& подпись студента\\
		& $\phantom{\int\limits_f(z)dz=0}$\\
		& Научный руководитель:\\
		& доктор физико-математических наук \\
		& Васенин Валерий Александрович\\
		& $\phantom{C_n^k=C_n^{n-k}}$\\
		& $\underline{\phantom{\int\limits_a^bf(x)dx=F(b)-F(a)}}$\\
		& подпись научного руководителя\\
	\end{tabular}
\end{center}
\vspace{1cm}
\begin{center}
	Москва\\
	$2022$
\end{center}

\newpage
\tableofcontents{}
\newpage	
	
 \section*{Тема}



\section{Введение}


\newpage
\section{Постановка задачи}

Пусть задан граф $G = (V, E)$, описывающий некоторую \textit{дорожную сеть}. Предположим, что имеется $n$ участников движения по этому графу. Каждый участник $i$ имеет точки отправления $A_i \in V$ и точки прибытия $B_i \in V$. Пусть множество $P_i$ - есть множество всех простых путей из $A_i$ в $B_i$. Пусть декартово произведение $P = \prod \limits_{i = 1} ^ n P_i$ есть множество всех возможных комбинаций путей участников. Элементы этого множества назовем \textit{комбинацией путей}. Пусть известно, что при комбинации путей участников $\textbf{p} \in P$ $i$-ый участник затрачивает $T_i(\textbf{p})$ времени на передвижение. Суммарные временные затраты положим $T(\textbf{p}) = \sum\limits_{i = 1}^n T_i (\textbf{p})$. Пару $(P, \{T_i\}_{i = 1} ^ n)$ назовем \textit{некооперативным передвижением} на графе $G$. Функции $T_i: P \rightarrow R_+$ назовем \textit{функцией временных затрат}.

Необходимо найти такую комбинацию путей участников $\textbf{p}^*$, что суммарные временные затраты на передвижение - минимальны
\begin{equation}
\label{eq:target_task_T} 
T (\textbf{p}^*) = \minn\limits_{ \textbf{p} \in P} T (\textbf{p})
\end{equation}
	
Комбинацию путей $\textbf{p}^*$ будем называть \textit {оптимальной}, а суммарные временные затраты  $ T (\textbf{p}^*)$ \textit {оптимальным временем передвижения участников}.

\newpage
\section{Построение функций временных затрат}

Сложность численного решения задачи поиска оптимальной комбинации путей во многом зависит от аналитического задания функций
$T_i (\textbf{p})$. Интуитивно вполне очевидно, что на временные затраты при проезде по пути $\textbf{p}_i$ в первую очередь влияют временные затраты на ребрах, составляющих маршрут $\textbf{p}_i$. Поэтому без ограничения общности считаем, что функции временных затрат есть суммарные временные затраты на каждом ребре этого пути

$$T_i (\textbf{p}) = \sum \limits_{e \in E} \overline{\tau}_{e, i} (\textbf{p}) = \sum \limits_{e \in \textbf{p}_i} \overline{\tau}_{e, i} (\textbf{p}), $$
где функции $\overline{\tau}_{e, i} (\textbf{p})$ есть временные затраты $i$-ого участника на ребре $e$ при комбинации путей $\textbf{p}$. Поскольку подразумевается, что передвижение участников происходит непрерывно во времени, то, можно считать, что временные затраты на ребре $e$ есть затраченное участником время на этом ребре

$$  \overline{\tau}_{e, i} (\textbf{p}) = \int\limits_{0}^{\infty} \theta_{e, i} (\textbf{p}, t) dt, $$
где 

$$
\theta_{e, i} (\textbf{p}, t) =
\begin{cases}
	1, & \text{если}  \text{ i-ый участник движется по ребру $e$ в момент времени $t$}  \\
	0, & \text{иначе}
\end{cases}
$$

Для простоты записи введем функцию, отвечающую за количество машин на ребре $e$ в момент времени $t$

$$n_e(\textbf{p}, t) = \sum \limits_{i = 1}^n \theta_{e, i} (\textbf{p}, t) = \{\text{количество машин на ребре } e \text { в момент времени } t\}$$

Таким образом, суммарные временные затраты есть 

\begin{equation}
\label{eq:target_func_n_e}
T(\textbf{p}) = \sum \limits_{e \in E} \int\limits_{0}^{\infty} n_e (\textbf{p}, t) dt
\end{equation}

Поскольку передвижение каждого участника проходит непрерывно, функции $\theta_{e, i} (\textbf{p}, t)$ являются индикаторами некоторых интервалов $[t_{e, i}^{in} (\textbf{p}), t_{e, i}^{out} (\textbf{p})]$. Также $\theta_{e, i} (\textbf{p}, t)$ описывают движение по некоторому простому пути $\textbf{p}_i$, поэтому стоит ввести ограничения на $t_{e, i}^{in} (\textbf{p}), t_{e, i}^{out} (\textbf{p})$:

\begin{equation}
\label{eq:restr_t}
 \begin{cases}
	\overline{\tau}_{e, i}^{min} \le t_{e, i}^{out}(\textbf{p}) - t_{e, i}^{in}(\textbf{p}) \le \overline{\tau}_{e, i}^{max}, e \in \textbf{p}_i, i = 1, \dots, n,
	\\
	t_{e, i}^{out}(\textbf{p}) - t_{e, i}^{in}(\textbf{p}) = 0, e \notin \textbf{p}_i, i = 1, \dots, n,
	\\
	 \sum \limits_{ \substack{E_1 = \{ e \in E: e = (X_1, B) \} \\ e \in E_1}} t_{e, i}^{out} (\textbf{p}) = \sum \limits_{ \substack{E_2 = \{ e \in E: e = (B, X_2) \} \\ e \in E_2}} t_{e, i}^{in} (\textbf{p}), B \in V, i = 1, \dots, n, E_1 \ne \emptyset, E_2 \ne \emptyset,
\end{cases}
\end{equation}

где конснтанты $\overline{\tau}_{e, i}^{min}$ и $\overline{\tau}_{e, i}^{max}$ - ограничения на функцию $\overline{\tau}_{e, i} (\textbf{p})$. Без ограничения общности считаем, что мы рассматриваем такое движение, что эти константы существуют (ограничено время проезда участника по ребру) и они положительны (нельзя пройти ребро за время $\overline{\tau}_{e, i} (\textbf{p})$ = 0).

Целевая функция в этом случае есть

\begin{equation}
	\label{eq:target_func}
		\sum \limits_{i = 1}^n \sum \limits_{e \in E} t_{e, i}^{out}(\textbf{p}) - t_{e, i}^{in}(\textbf{p}) \rightarrow \minn \limits _{\textbf{p} \in P}
\end{equation}

Заметим, что система \eqref{eq:restr_t}, \eqref{eq:target_func} эквивалентна задаче смешнного целочисленного линейного программирования. Однако, в такой постановке задача эквивалентна задаче поиска кратчайшего пути для каждой машины $i$ между вершинами $A_i$ в $B_i$ с весами $\overline{\tau}_{e, i}^{min}$. На практике машины, находящиеся вблизи друг друга влияют на скорости друг друга.

Для того, чтобы учесть влияние участников друг на друга, для каждого участника $i$ введем микроскопическую характеристику движения $v_i(\textbf{p}, t)$, описывающую скорость участника.
Тогда, имеет место

\begin{equation}
\label{eq:velocity_eq}
\int\limits_{0}^{\infty} \theta_{e, i} (\textbf{p}, t) v_i(\textbf{p}, t) dt = l_e, e \in \textbf{p}_i, i = 1, \dots, n,
\end{equation}

где $l_e$ - длина ребра $e \in E$. Будем говорить, что уравнения \eqref{eq:velocity_eq} задают \textit{модель движения участников}. Таким образом задача \eqref{eq:target_task_T} эквивалентна следующей задаче оптимизации

\begin{equation}
\label{eq:target_task_theta_v}
\begin{cases}
	\sum \limits_{i = 1}^n \sum \limits_{e \in E} t_{e, i}^{out}(\textbf{p}) - t_{e, i}^{in}(\textbf{p}) \rightarrow \minn \limits _{\textbf{p} \in P}
	\\
	\overline{\tau}_{e, i}^{min} \le t_{e, i}^{out}(\textbf{p}) - t_{e, i}^{in}(\textbf{p}) \le \overline{\tau}_{e, i}^{max}, e \in \textbf{p}_i, i = 1, \dots, n,
	\\
	t_{e, i}^{out}(\textbf{p}) - t_{e, i}^{in}(\textbf{p}) = 0, e \notin \textbf{p}_i, i = 1, \dots, n,
	\\
	\sum \limits_{ \substack{E_1 = \{ e \in E: e = (X_1, B) \} \\ e \in E_1}} t_{e, i}^{out} (\textbf{p}) = \sum \limits_{ \substack{E_2 = \{ e \in E: e = (B, X_2) \} \\ e \in E_2}} t_{e, i}^{in} (\textbf{p}), B \in V, i = 1, \dots, n, E_1 \ne \emptyset, E_2 \ne \emptyset,
	\\
	\int\limits_{0}^{\infty} \theta_{e, i} (\textbf{p}, t) v_i(\textbf{p}, t) dt = l_e, e \in \textbf{p}_i
\end{cases}
\end{equation}

Заметим, что в случае, когда условие \eqref{eq:velocity_eq} можно описать в виде задачи смешанного целочисленного программирования, 
задача \eqref{eq:target_task_theta_v} может быть решена стандартным решателем.

\newpage
\section{Модели движения}

\subsection*{Макроскопические модели}

Предположим скорость участника зависит только загруженности ребра, на котором он находится:

\begin{equation}
	\label{eq:velocity_eq_macro}
	 v_i(\textbf{p}, t) = \sum \limits _{e \in E} \theta_{e, i} (\textbf{p}, t) v (n_e (\textbf{p}, t)),  i = 1, \dots, n
\end{equation}

Такую модель движения в дальнейшем будем называть \textit{макроскопической}.
Например, естествено расмотреть модель $ v (n_e (\textbf{p}, t)) = \frac{v_{max}}{n_e (\textbf{p}, t)}$

\begin{lemma}
	\label{lemma:lt}
	Пусть даны переменные $a$, $b$ целочисленного программирования и известно, что существует $M > 0$ : $|a| < M$, $|b| < M$. Тогда можно добавить новую целочисленную переменную $\textbf{1} (\{a < b\}) \in \{0, 1\}$ такую, что
	
	\begin{equation*}
		\textbf{1} (\{a < b\}) = 
		\begin{cases}
			1, a < b,
			\\
			0, a \ge b
		\end{cases}
	\end{equation*}

\end{lemma}

\begin{proof}
	Добавим в нашу задачу два неравенства:
	
	$$ 2M (\textbf{1} (\{a < b\}) - 1) < b - a \le 2M\textbf{1} (\{a < b\}) $$
	
	Очевидная проверка показывает, что неравенство выполняется для любых $a, b$.
	
	
\end{proof}

\begin{state}
	
	\label{state:lin_prog}
	
	Пусть модель движения $ v_i(\textbf{p}, t)$ макроскопическая. Тогда задача \eqref{eq:target_task_theta_v} есть задача смешанного целочисленного линейного программирования.
\end{state}

\begin{proof}
	Докажем для случая $n = 2$. Для случаев $n \ge 2$ доказательство аналогичное.
	
	Пусть имеется задача смешанного целочисленного линейного программирования \eqref{eq:restr_t} с переменными $t_{e, i}^{in}, t_{e, i}^{out}, I_{e, i}, e \in E, i = 1, 2$. Преобразуем условие \eqref{eq:velocity_eq} к каноническому виду. Для удобства обозначим обоих участников индексами $i, j \in \{1, 2\}$.
	
	$$\int\limits_{0}^{\infty} \theta_{e, i} (\textbf{p}, t) v_i(\textbf{p}, t)dt = \int\limits_{0}^{\infty} \theta_{e, i} (\textbf{p}, t) \sum \limits _{e^1 \in E} \theta_{e^1, i} (\textbf{p}, t) v (n_{e^1} (\textbf{p}, t)) dt = $$
	
	$$\int\limits_{0}^{\infty} \theta_{e, i} (\textbf{p}, t)  v (n_{e} (\textbf{p}, t)) dt = 
	  \int\limits_{ \substack{n_{e} (\textbf{p}, t) = 1}} \theta_{e, i} (\textbf{p}, t)  v (n_{e} (\textbf{p}, t)) dt +
	  \int\limits_{ \substack{n_{e} (\textbf{p}, t) = 2}} \theta_{e, i} (\textbf{p}, t)  v (n_{e} (\textbf{p}, t)) dt = $$
	
	$$\int\limits_{ \substack{\theta_{e, j} (\textbf{p}, t) = 0}} \theta_{e, i} (\textbf{p}, t)  v (1) dt +
	  \int\limits_{ \substack{\theta_{e, j} (\textbf{p}, t) = 1}} \theta_{e, i} (\textbf{p}, t)  v (2) dt = $$
	  
    $$\int\limits_{0}^{\infty} \theta_{e, i} (\textbf{p}, t)  v (1) dt - 
      \int\limits_{ \substack{\theta_{e, j} (\textbf{p}, t) = 1}} \theta_{e, i} (\textbf{p}, t)  v (1) dt +
	  \int\limits_{ \substack{\theta_{e, j} (\textbf{p}, t) = 1}} \theta_{e, i} (\textbf{p}, t)  v (2) dt = $$
	  
	$$v (1) \int\limits_{0}^{\infty} \theta_{e, i} (\textbf{p}, t) dt +
	  (v (2) - v(1)) \int\limits_{0}^{\infty} \theta_{e, i} (\textbf{p}, t) \theta_{e, j} (\textbf{p}, t) dt = $$
	  
    $$v (1) \overline{\tau}_{e, i} (\textbf{p}) +
    (v (2) - v(1)) \int\limits_{0}^{\infty} \theta_{e, i} (\textbf{p}, t) \theta_{e, j} (\textbf{p}, t) dt = l_e, e \in \textbf{p}_i$$
	  
	Неизвестный интеграл - время совместного проезда участников на ребре $e$.
	  
	В переменных задачи смешанного целочисленного программирования получим:
	
	$$v(1) (t_{e, i}^{out} - t_{e, i}^{in}) + (v(2) - v(1)) (t_{e, ij}^{out} - t_{e, ij}^{in}) = l_e I_{e, i},$$
	
	где новые перемнные $t_{e, ij}^{in}$, $t_{e, ij}^{out}$ отвечают за начало и конец совместного проезда участников. Просуммировав по всем ребрам $e \in E$, получим
	
	$$v(1) \sum \limits _{e \in E} (t_{e, i}^{out} - t_{e, i}^{in}) = \sum \limits _{e \in E} l_e I_{e, i} - (v(2) - v(1)) \sum \limits _{e \in E} (t_{e, ij}^{out} - t_{e, ij}^{in})$$
	
	Заметим, что левая часть есть временные затраты участника $i$, поэтому задачу оптимизации можно переписать в виде
	$$ \frac{1}{v (1)} \sum\limits_{i = 1}^n \sum \limits _{e \in E} l_e I_{e, i} + \frac{v(1) - v(2)}{v (1)}  \sum\limits_{i = 1}^n \sum \limits _{e \in E} (t_{e, ij}^{out} - t_{e, ij}^{in}) \rightarrow \minn $$
	
	Для завершения доказательства необходимо показать, что переменные $t_{e, ij}^{in}, t_{e, ij}^{out}$ описываются линейными ограничениями. Напомним, что $[t_{e, ij}^{in}, t_{e, ij}^{out}] = [t_{e, i}^{in}, t_{e, i}^{out}] \cap [t_{e, j}^{in}, t_{e, j}^{out}]$. Обозначим $\Delta t = t_{e, ij}^{out} - t_{e, ij}^{in}, \Delta t_1 =  t_{e, i}^{out} - t_{e, i}^{in}, \Delta t_2 =  t_{e, j}^{out} - t_{e, j}^{in}, \Delta t_3 =  t_{e, i}^{out} - t_{e, j}^{in}, \Delta t_4 =  t_{e, j}^{out} - t_{e, i}^{in}$
	
	Используя лемму \ref{lemma:lt}, при $M = max (\overline{\tau}_{e, i}^{max}, \overline{\tau}_{e, j}^{max})$, добавим в задачу новые переменные $\textbf{1} (\{ \Delta t_k > \Delta t_l\}), k \ne l, k, l \in {1, 2, 3, 4}$. Рассмотрим велечину $T_{max} = \sum \limits_{e \in E} max (\overline{\tau}_{e, i}^{max}, \overline{\tau}_{e, j}^{max})$. Добавим в нашу задачу следующие неравенства: 
	
	$$\Delta t \ge 0, $$
	$$\Delta t \ge \Delta t_k - T_{max} \sum \limits_{l != k} {\textbf{1} (\{ \Delta t_k > \Delta t_l\})}, k = 1, 2, 3, 4$$
	
	Тогда переменная $\Delta t$ есть длина отрезка $[t_{e, ij}^{in}, t_{e, ij}^{out}]$. 
	
\end{proof}

\begin{corollary}
	\label{corollary:rel}

	Пусть модель движения $ v_i(\textbf{p}, t) = \sum \limits _{e \in E} \theta_{e, i} (\textbf{p}, t) v (n_e (\textbf{p}, t))$ макроскопическая и последовательность $v(n) > 0, \forall n \in \mathbb{Z}_+$ убывает. Предположим, что оптимальное время движения в модели c постоянными скоростями $v(1)$ есть $\widetilde{T}$. Тогда имеет место

	$$ \frac{2 v(1) - v(n)}{v(1)} \widetilde{T} \le T \le \frac {v(1)}{v(n)} \widetilde{T}$$
	
\end{corollary}

\begin{proof}
	Докажем каждое неравенство в отдельности
	
	1. В модели, где все участники едут с постоянными скоростям движение происходит по кратчайшим путям. Тогда временные затраты есть $\widetilde{T} = \frac{1}{v(1)} \sum \limits _{i = 1} ^ n \sum\limits_{e \in p_i} l_e$, где $p_i$ - кратчайшие пути.
	На тех же путях задается самый худший случай макроскопической модели - все едут с минимальной скоростью, то есть $ T = \frac{1}{v(n)} \sum \limits _{i = 1} ^ n \sum\limits_{e \in p_i} l_e$. Тогда получим
	
	$$T \le  \frac{1}{v(n)} \sum \limits _{i = 1} ^ n \sum\limits_{e \in p_i} l_e = \frac {v(1)}{v(n)} \widetilde{T}$$
	
	2. Проделывая аналогичне выкладки что и в доказательстве \ref{state:lin_prog}, можно получить, что функция оптимизация есть 
	
	$$ \frac{1}{v (1)} \sum\limits_{i = 1}^n \sum \limits _{e \in E} l_e I_{e, i} +  \sum\limits_{k = 2}^{n} \frac{v(1) - v(k)}{v (1)}  \sum\limits_{i = 1}^n \sum \limits _{e \in E} \sum\limits _{\substack{ s_k \in 2^n \\ |s_k| = k}}  \Delta t_{e, s_k} \rightarrow \minn ,$$
	
	где имеются ограничения
    
    $$I_{e, i} \frac{l_e}{v(1)} \le  \sum\limits _{\substack{ s_k \in 2^n \\ i \in s_k}}  \Delta t_{e, s_k} \le I_{e, i} \frac{l_e}{v(n)} , e \in E,  i = 1, \dots, n$$
	
	Тогда получим
		$$T \ge 
		  \minn \left(  \frac{1}{v (1)} \sum\limits_{i = 1}^n \sum \limits _{e \in E} l_e I_{e, i} \right) 
		+ \minn \left(  \sum\limits_{k = 2}^{n} \frac{v(1) - v(k)}{v (1)}  \sum\limits_{i = 1}^n \sum \limits _{e \in E} \sum\limits _{\substack{ s_k \in 2^n \\ |s_k| = k}}  \Delta t_{e, s_k} \right) \ge $$
		
		$$\ge \widetilde{T} + \frac{v(1) - v(n)}{v(1)} \widetilde{T} =  \frac{2 v(1) - v(n)}{v(1)} \widetilde{T} $$

\end{proof}

\subsection*{Микроскопические модели}
\textit {Микроскопическими} называются модели, в которых явно исследуется движение каждого автомобиля.
Выбор такой модели позволяет теоретически достичь более точного описания движения автомобилей по сравнению с макроскопической моделью, однако этот подход требует больших вычислительных ресурсов при практических применениях.

Например, на практике скорость автомобиля напрямую зависит от скорости и положения автомобиля спереди. 

Для простоты рассмотрим однополосное бесконечное движение. Пусть $x_i(t) \in [0, +\infty)$ - координаты на полосе участника $i$. Предположим, что скорость участника ограничена некоторой общей велечиной $v_{max}$. Пусть в момент времени $t = 0$ выполняется $x_1(0) \le х_2(0) \le \dots \le x_n(0)$.

\subsubsection*{Модель пропорциональной скорости}
Рассмотрим пример, когда скорость машины пропорциональна расстоянию до следующей машины.
Для удобства положим $d_{i} (t) = x_{i + 1} (t) - x_{i} (t), i = 1, \dots, n - 1$.
Без ограничения общности считаем, что  $d_{i} (0) < D$, где $D$ - характерное расстояние взаимодействия участников.

Пусть модель движения есть
\begin{equation}
	\label{eq:micro}
	v_i(t)=
	\begin{cases}
		v_{max}, & i = n
		\\
		v_{max} \frac{d_i(t)}{D} ,& i \ne n
	\end{cases}
\end{equation}

Для поиска функций $x_i(t)$ достаточно рассмотреть систему дифференциальных уравнений

$$ \dot{d_i} (t) = v_{i + 1} (t) - v_i (t) $$

Решением такой системы является

$$d_{n - k} (\tau) = \sum \limits_{l = 0} ^ {k - 1} \left(\frac{d_{n - k + l} (0) - D}{l!} \tau^l e ^ {-\tau}\right) + D ,$$

где $\tau = \frac{v_{max}}{D}t$. Модель обладает тем свойством, что порядок участников постоянен и участники не покидают зону взаимодействия $D$. 

Данная модель хорошо описывает реальное движение участников, однако ее практическое применение вызывает сложности, поскольку решение уравнения $x_i (t) = x_0$ может быть найдено только приближенно.

\subsubsection*{Модель снижения скорости}

Предположим, что существует некоторая велечина $c_n$, которая отвечает за снижение скорости хвоста скопления участников:

$$v_{n - k} = v_{max} - c_n k, \text{  } k = 0, \dots, n - 1$$

Велечину $c_n$ выберем из соображений, что $v_0 = \frac{v_{max}}{n}$. Тогда $c_n = \frac{v_{max}}{n}$. Если исследовать данную модель на графе, то функция скоростей будут кусочно постоянными. Это связано с тем, что некоторые участники сьезжают с пути друг друга или покидают характерное расстояние взаимодействия. Поэтому она не лучшим образом описывает реальное движение, однако проста в использовании.

\end{document}