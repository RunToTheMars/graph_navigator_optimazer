\documentclass{beamer}

\usepackage[utf8]{inputenc}
\usepackage[T2A]{fontenc}
\usepackage[english,russian]{babel}


\usepackage{tikz}
\usetikzlibrary{arrows,decorations.pathmorphing,backgrounds,fit,positioning,shapes,shapes.symbols,chains}

\usetheme{Frankfurt}

% Переведем заголовки блоков на русский
\uselanguage{russian}
\languagepath{russian}
\deftranslation[to=russian]{Theorem}{Теорема}
\deftranslation[to=russian]{Example}{Пример}


% \usefonttheme{professionalfonts}
\usefonttheme{serif}
% \usefonttheme{structureitalicserif}


\DeclareMathOperator*{\minn}{min}
\DeclareMathOperator*{\argmin}{argmin}

\begin{document}

\title{Оптимизация транспортного потока при заданных пунктах отправления и назначения всех участников движения}
\author{Пехтерев С.И. 610 группа\\Научный руководитель: д.ф.-м.н. Васенин В.А.}
\institute[]{}
\date[15.05.2022]{15 мая 2022}

% Создание заглавной страницы
% \frame{\titlepage}
\maketitle


% % Автоматическая генерация содержания
% \frame{
%   \frametitle{План}
%   \tableofcontents
% }

\section{Задача кооперативной оптимизации транспортного потока}

\begin{frame}\frametitle{Основные определения}
\emph{Дорожной сетью} назовем тройку $G = (V, E, l)$, где $(V, E)$ --- ориентированный граф с длинами ребер $l: E \rightarrow \mathbb{R}_{>0} $.  Предположим, что имеется $n$ участников с заданными точками отправления $A_i \in V$ и прибытия $B_i \in V$. Пусть множество $P_i$ есть множество всех простых путей из $A_i$ в $B_i$. Элемент декартового произведения ${P = \prod \limits_{i = 1} ^ n P_i}$ назовем \emph{комбинацией путей}. Пусть известно, что при комбинации путей участников $\textbf{p} = \left(p_1, \ldots, p_n\right)\in P$ $i$-ый участник затрачивает $T_i(\textbf{p}) \in \mathbb{R}_{\ge 0}$ времени на свой путь. 
Функции $T_i$ назовем \textit{функциями временных затрат} участника $i$.
\end{frame}

\begin{frame}\frametitle{Некооперативное прокладывание пути}
  
  \textit{Некооперативным прокладыванием пути} в дорожной сети $G$ назовем пятерку $F = (n, G, \{A_i\}_{i = 1}^{n}, \{B_i\}_{i = 1}^{n}, \{T_i\}_{i = 1}^{n})$. Некооперативное прокладывание пути предполагает, что каждый участник стремится сократить собственные временные затраты выбором пути $p_i$, несмотря на временные затраты других участников. 
\end{frame}

\begin{frame}\frametitle{Задача прокладывания набора путей}
  Введем некоторую функцию $\Phi (\textbf{p}) = \phi (T_1 (\textbf{p}), \ldots, T_n(\textbf{p}))$, определенную на множестве всех возможных комбинаций путей $P$ и отображающую его во множество действительных чисел. С помощью нее участники могут отслеживать, как влияет изменение их пути на общую картину движения. Такую функцию назовем \textit{функцией стоимости}.

Для заданных некооперативного прокладывания пути $F$ и функции стоимости $\Phi$ необходимо найти комбинацию путей $\textbf{p}^*$ такую, что функция стоимости на ней минимальна, то есть
\begin{equation}
  \Phi (\textbf{p}^*) = \minn\limits_{ \textbf{p} \in P} \Phi (\textbf{p}).
\end{equation}
  
\end{frame}


\section{Модель движения}

\begin{frame}\frametitle{}
\textit{Модель движения} $v_i(\textbf{p}, t)$ - положительная функция, отделенная от нуля ограниченная функция, для которой верно


\end{frame}


\begin{frame}
Спасибо за внимание!
\end{frame}


\end{document}
