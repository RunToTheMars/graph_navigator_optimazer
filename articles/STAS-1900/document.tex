\documentclass[12pt, a4paper]{article}
\usepackage[utf8]{inputenc}
\usepackage[russian]{babel}
\usepackage[T2A]{fontenc}
\usepackage{amsfonts}

\usepackage[left=2cm,right=1.5cm,top=2cm,bottom=2cm]{geometry}
\linespread{1.25}

\usepackage{graphicx}
\graphicspath{{pictures/}}
\DeclareGraphicsExtensions{.pdf,.png,.jpg}

\begin{document}
	\section*{Тема}
	Оптимизация транспортного потока при заданных пунктах отправления и назначения всех участников движения
	
	\section*{Введение}
	
	Данная дипломная работа посвящена одной из задач математического моделировнаия транспортных потоков. А именно нас интересует построение маршрутов при заданных координатах начала и конца движения на фиксированной карте дорожной сети. Целью является нахождение <<оптимальных>> путей, хотя выбор критерия оптимальности достаточно широк и неоднозначен, ведь участники влияют друг на друга и что хорошо для одного, может критически отразиться на движении другого. Мы предполагаем, что изначально дорожная сеть не содержит движущихся автомобильных транспортных (автотранспортных) средств (АТС) и наша задача, на самом деле, является задачей \textit{управления} транспортными потоками -- мы задаем направление движения каждого участника в каждый момент времени.
	
	
	
	Говоря об актуальности задачи, достаточно сказать, что на данный момент существует множество научных журналов\footnote{Перечень научных журналов: Transportation Research B, Physical Review E, Review of modern physics, Transportation Science. Электронные ресурсы -- например, http://arxiv.org/}, в которых регулярно публикуются статьи на транспортную тематику. Также известное немецкое издательство Springer публикует труды ученых, представленных на конференции по математическому моделированию транспортных потоков «Traffic and granular flow», которая проводится с периодичностью в 2 года. Упомянутая конференция проходила и в Москве в 2011 году\footnote{На сайте https://link.springer.com/ можно ознакомиться с программой конференции.}. Также в столице регулярно проводится семинар\footnote{На следующих электронных ресурсах http://kozlov-traffic-ras.ru/, http://wtran.dvo.ru/ можно познакомиться с работами участников этого семинара.} «Научно-практические задачи развития автомобильно-дорожного комплекса в России» под руководством вицепрезидента РАН акад. В. В. Козлова.
	
	Данная тема очень широка и включает в себя множество задач, таких как эволюция затора, задача о светофоре, задача о надежности графа транспортной сети и др. Наша задача затрагивает другие проблемы и входит в класс задач транспортного равновесия. Моделирование как задача принятия решений позволяет получить прогнозные оценки по загрузке элементов транспортной сети. Поставленная нами задача интересна тем, что она может служить инструментом для оценки эффективности проектов по модификации улично-дорожных сетей (УДС) с точки зрения разгрузки наиболее проблемных участков дорог и уменьшения общих затрат на передвижение пользователей сети.
	
	На основе транспортного равновесия базируется множество научных работ. В одной из них рассматривается модель Бэкмана и поиск равновеися в этой модели методом Франк–Вульфа \cite{litlink1}--\cite{litlink4}. Данная модель имеет ряд ограничений и допущений, вследствие в качестве замены была предложена модель стабильной динамики, которая наследует все преимущества модели Бэкмана и устраняет часть ее недостатков \cite{litlink5}--\cite{litlink7}. ...
	
	Руководствуясь определенными принципами в теории транспортного равновесия, мы попробуем решить задачу построения оптимальных маршрутов итерационными методами, такими-то такими-то методами, не знаю, какими. 
	
	Более подробно о постановке задачи вы сможете узнать в следующей главе, после чего следует описание возможных решений. В третьей главе мы посчитаем их сложность, зафиксируем положительные результаты исследования и проанализируем недостатки, подумаем над будущими улучшениями и другими решениями задачи. В конце оценим качество проделанной работы и подытожим результаты.
	
	
	\newpage
	
	% даём указание на включение данного место в оглавление как секции (\section)
	\addcontentsline{toc}{section}{Список используемой литературы}
	
	%далее сам список используевой литературы
	\begin{thebibliography}{}
		\bibitem{litlink1}  Beckmann M., McGuire C.B., Winsten C.B. Studies in the economics of transportation. RM1488. Santa Monica: RAND Corporation, 1955.
		\bibitem{litlink2}  Sheffi Y. Urban transportation networks: Equilibrium analysis with mathematical programming methods. N.J.: Prentice–Hall Inc., Englewood Cliffs, 1985.
		\bibitem{litlink3}  Frank M., Wolfe P. An algorithm for quadratic programming //
		Naval Research Logistics Quarterly. 1956. V. 3. P. 95–110.
		\bibitem{litlink4}  А. В. Гасников, П. Е. Двуреченский, Ю. В. Дорн, Ю. В. Максимов, Численные методы поиска равновесного распределения потоков в модели Бэкмана и в модели стабильной динамики, Матем. моделирование, 2016, том 28, номер 10, 40–64
		\bibitem{litlink5} Nesterov Y., de Palma A. Stationary Dynamic Solutions in Congested Transportation Networks: Summary and Perspectives // Networks Spatial Econ. 2003. № 3(3). P. 371–395.
		\bibitem{litlink6} Nesterov Y. Stable traffic equilibria: properties and applications // Optimization and Engineering. 2000. V.1.P.29-50
		\bibitem{litlink7} Гасников А.В., Дорн Ю.В., Нестеров Ю.Е, Шпирко С.В. О трехстадийной версии модели
		стационарной динамики транспортных потоков // Математическое моделирование. 2014. Т. 26:6. C. 34–70.
	\end{thebibliography}
	
	
\end{document}